% JSON-LD Context and Semantic Web Integration
% Documentation for BOOST JSON-LD support

BOOST implements JSON-LD (JSON for Linking Data) as its primary serialization format, enabling semantic web compatibility, data linking, and machine-readable context definitions. This section explains the JSON-LD context structure, semantic annotations, and integration with existing ontologies.

\subsection{JSON-LD Overview}
\label{sec:jsonld-overview}

JSON-LD extends standard JSON with semantic web capabilities through:

\begin{itemize}
    \item \textbf{@context}: Defines mappings between JSON properties and RDF vocabularies
    \item \textbf{@id}: Provides unique identifiers for entities (IRIs)
    \item \textbf{@type}: Specifies the semantic type of an entity
    \item \textbf{@vocab}: Sets a default vocabulary for properties
    \item \textbf{Linked Data}: Enables connections between distributed datasets
\end{itemize}

\subsection{BOOST Context Definition}
\label{sec:boost-context}

The BOOST JSON-LD context maps entity properties to established vocabularies:

\begin{jsonexample}{BOOST Core Context}
\begin{minted}[fontsize=\small,linenos=false,breaklines=true,tabsize=2]{json}
{
  "@context": {
    "schema": "http://schema.org/",
    "prov": "http://www.w3.org/ns/prov#",
    "gs1": "https://gs1.org/voc/",
    "biomass": "http://example.org/biomass#",
    "geo": "http://www.w3.org/2003/01/geo/wgs84_pos#",
    "qudt": "http://qudt.org/schema/qudt/",
    "unit": "http://qudt.org/vocab/unit/",
    
    "TraceableUnit": "biomass:TraceableUnit",
    "Organization": "schema:Organization",
    "Transaction": "schema:Order",
    "MaterialProcessing": "prov:Activity",
    
    "traceableUnitId": {
      "@id": "schema:identifier",
      "@type": "schema:Text"
    },
    "organizationId": {
      "@id": "schema:identifier",
      "@type": "schema:Text"
    },
    "createdAt": {
      "@id": "schema:dateCreated",
      "@type": "xsd:dateTime"
    },
    "modifiedAt": {
      "@id": "schema:dateModified",
      "@type": "xsd:dateTime"
    }
  }
}
\end{minted}
\end{jsonexample}

\subsection{Vocabulary Mappings}
\label{sec:vocabulary-mappings}

\subsubsection{Schema.org Integration}

BOOST entities map to Schema.org types for web compatibility:

\begin{itemize}
    \item \textbf{Organization} → \texttt{schema:Organization}
    \item \textbf{Transaction} → \texttt{schema:Order}
    \item \textbf{GeographicData} → \texttt{schema:Place}
    \item \textbf{Certificate} → \texttt{schema:Certification}
    \item \textbf{Claim} → \texttt{schema:Claim}
\end{itemize}

\subsubsection{W3C PROV Ontology}

Provenance tracking using PROV vocabulary:

\begin{itemize}
    \item \textbf{MaterialProcessing} → \texttt{prov:Activity}
    \item \textbf{ProcessingHistory} → \texttt{prov:Entity}
    \item \textbf{Operator} → \texttt{prov:Agent}
    \item \textbf{wasGeneratedBy} → \texttt{prov:wasGeneratedBy}
    \item \textbf{wasAttributedTo} → \texttt{prov:wasAttributedTo}
\end{itemize}

\subsubsection{GS1 Vocabulary}

Supply chain standards alignment:

\begin{itemize}
    \item \textbf{productCode} → \texttt{gs1:gtin}
    \item \textbf{locationCode} → \texttt{gs1:gln}
    \item \textbf{shipmentId} → \texttt{gs1:sscc}
    \item \textbf{batchNumber} → \texttt{gs1:batchNumber}
\end{itemize}

\subsection{Entity Context Examples}
\label{sec:entity-context-examples}

\subsubsection{TraceableUnit with Context}

Complete JSON-LD representation of a TraceableUnit:

\begin{jsonexample}{TraceableUnit JSON-LD}
\begin{minted}[fontsize=\small,linenos=false,breaklines=true,tabsize=2]{json}
{
  "@context": "https://boost.org/context.jsonld",
  "@type": "biomass:TraceableUnit",
  "@id": "https://example.org/tru/TRU-2025-001",
  
  "traceableUnitId": "TRU-2025-001",
  "unitType": "pile",
  "totalVolume": {
    "@type": "qudt:QuantityValue",
    "qudt:value": 500.0,
    "qudt:unit": "unit:M3"
  },
  "speciesComposition": [{
    "@type": "biomass:SpeciesComponent",
    "species": "Pseudotsuga menziesii",
    "percentage": 75.0
  }],
  "harvestLocation": {
    "@type": "geo:Point",
    "geo:lat": 45.5231,
    "geo:long": -122.6765
  },
  "prov:wasGeneratedBy": {
    "@id": "https://example.org/harvest/HARV-2025-001"
  },
  "prov:wasAttributedTo": {
    "@id": "https://example.org/org/ORG-FOREST-001"
  }
}
\end{minted}
\end{jsonexample}

\subsubsection{Transaction with Linked Data}

Transaction linking multiple entities:

\begin{jsonexample}{Linked Transaction}
\begin{minted}[fontsize=\small,linenos=false,breaklines=true,tabsize=2]{json}
{
  "@context": "https://boost.org/context.jsonld",
  "@type": "schema:Order",
  "@id": "https://example.org/txn/TXN-2025-001",
  
  "transactionId": "TXN-2025-001",
  "schema:seller": {
    "@id": "https://example.org/org/ORG-SUPPLIER-001"
  },
  "schema:buyer": {
    "@id": "https://example.org/org/ORG-BUYER-001"
  },
  "schema:orderedItem": [{
    "@id": "https://example.org/tru/TRU-2025-001"
  }],
  "schema:price": {
    "@type": "schema:PriceSpecification",
    "schema:price": 85.50,
    "schema:priceCurrency": "USD"
  },
  "prov:startedAtTime": "2025-01-15T09:00:00Z",
  "prov:endedAtTime": "2025-01-15T14:30:00Z"
}
\end{minted}
\end{jsonexample}

\subsection{Advanced Features}
\label{sec:jsonld-advanced}

\subsubsection{Named Graphs}

Support for multi-source data using named graphs:

\begin{jsonexample}{Named Graph Structure}
\begin{minted}[fontsize=\small,linenos=false,breaklines=true,tabsize=2]{json}
{
  "@context": "https://boost.org/context.jsonld",
  "@graph": [{
    "@id": "https://example.org/graph/supplier",
    "@graph": [
      {
        "@type": "Organization",
        "organizationId": "ORG-001",
        "name": "Forest Products Inc"
      }
    ]
  }, {
    "@id": "https://example.org/graph/certification",
    "@graph": [
      {
        "@type": "Certificate",
        "certificateId": "CERT-FSC-001",
        "issuedTo": {"@id": "ORG-001"}
      }
    ]
  }]
}
\end{minted}
\end{jsonexample}

\subsubsection{Framing}

JSON-LD framing for specific data views:

\begin{jsonexample}{Frame Definition}
\begin{minted}[fontsize=\small,linenos=false,breaklines=true,tabsize=2]{json}
{
  "@context": "https://boost.org/context.jsonld",
  "@type": "TraceableUnit",
  "harvestedBy": {
    "@type": "Organization",
    "certifications": {
      "@type": "Certificate",
      "certificationType": "FSC"
    }
  }
}
\end{minted}
\end{jsonexample}

\subsubsection{Compaction and Expansion}

BOOST supports JSON-LD algorithms:

\begin{itemize}
    \item \textbf{Compaction}: Shortens IRIs using context
    \item \textbf{Expansion}: Expands to full IRIs
    \item \textbf{Flattening}: Creates flat graph structure
    \item \textbf{Normalization}: Canonical RDF representation
\end{itemize}

\subsection{Context Negotiation}
\label{sec:context-negotiation}

\subsubsection{Content Type Headers}

HTTP content negotiation support:

\begin{itemize}
    \item \texttt{application/ld+json} - JSON-LD format
    \item \texttt{application/json} - Plain JSON (context link in header)
    \item \texttt{text/turtle} - RDF Turtle format
    \item \texttt{application/n-quads} - N-Quads format
\end{itemize}

\subsubsection{Profile Parameters}

Profile-based context selection:

\begin{verbatim}
Accept: application/ld+json; 
        profile="https://boost.org/profiles/extended"
\end{verbatim}

\subsection{Implementation Guidance}
\label{sec:jsonld-implementation}

\subsubsection{Python Implementation}

Using PyLD library for JSON-LD processing:

\begin{pythonexample}{JSON-LD Processing}
\begin{minted}[fontsize=\small,linenos=false,breaklines=true]{python}
from pyld import jsonld
import json

# Load BOOST context
with open('boost_context.jsonld') as f:
    context = json.load(f)

# Create entity with context
tru = {
    "@context": context,
    "@type": "TraceableUnit",
    "traceableUnitId": "TRU-001",
    "totalVolume": 100.0
}

# Expand to full IRIs
expanded = jsonld.expand(tru)

# Compact with custom context
compacted = jsonld.compact(expanded, context)

# Convert to RDF
rdf = jsonld.to_rdf(tru)

# Frame for specific view
frame = {"@type": "TraceableUnit"}
framed = jsonld.frame(tru, frame)
\end{minted}
\end{pythonexample}

\subsubsection{JavaScript Implementation}

Browser and Node.js support:

\begin{jsonexample}{JavaScript JSON-LD}
\begin{minted}[fontsize=\small,linenos=false,breaklines=true,tabsize=2]{javascript}
const jsonld = require('jsonld');

// Process BOOST data
async function processBoostData(data) {
  // Add context
  data['@context'] = 'https://boost.org/context.jsonld';
  
  // Validate structure
  const expanded = await jsonld.expand(data);
  
  // Generate RDF
  const nquads = await jsonld.toRDF(data, {format: 'N-Quads'});
  
  return nquads;
}
\end{minted}
\end{jsonexample}

\subsection{Semantic Validation}
\label{sec:semantic-validation}

\subsubsection{SHACL Constraints}

Shape validation for semantic correctness:

\begin{jsonexample}{SHACL Shape}
\begin{minted}[fontsize=\small,linenos=false,breaklines=true,tabsize=2]{json}
{
  "@context": {"sh": "http://www.w3.org/ns/shacl#"},
  "@type": "sh:NodeShape",
  "sh:targetClass": "biomass:TraceableUnit",
  "sh:property": [{
    "sh:path": "biomass:totalVolume",
    "sh:datatype": "xsd:decimal",
    "sh:minInclusive": 0,
    "sh:maxInclusive": 10000
  }]
}
\end{minted}
\end{jsonexample}

\subsubsection{Reasoning and Inference}

Automatic inference capabilities:

\begin{itemize}
    \item Type inheritance from parent classes
    \item Property domain/range validation
    \item Transitive relationship discovery
    \item Consistency checking
\end{itemize}

\subsection{Benefits and Use Cases}
\label{sec:jsonld-benefits}

\subsubsection{Interoperability Benefits}

\begin{itemize}
    \item \textbf{Global Identifiers}: IRIs enable worldwide unique identification
    \item \textbf{Vocabulary Reuse}: Leverage existing ontologies
    \item \textbf{Tool Ecosystem}: Compatible with RDF/SPARQL tools
    \item \textbf{Web Integration}: SEO and knowledge graph inclusion
\end{itemize}

\subsubsection{Supply Chain Use Cases}

\begin{itemize}
    \item \textbf{Cross-Organization Linking}: Connect data across partners
    \item \textbf{Provenance Tracking}: Complete chain of custody
    \item \textbf{Regulatory Reporting}: Machine-readable compliance data
    \item \textbf{Certification Verification}: Linked certificate validation
\end{itemize}

The JSON-LD context provides BOOST with semantic web capabilities essential for modern supply chain interoperability and regulatory compliance.