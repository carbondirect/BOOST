% Introduction Section for BOOST Specification
% Converted from includes/introduction.inc.md

The Biomass Open-Source Traceability (BOOST) data standard defines a comprehensive, interoperable framework for tracking biomass materials through complex supply chains. BOOST enables transparent, verifiable, and consistent data exchange to support sustainability verification, regulatory compliance, and supply chain integrity across the biomass economy.

\subsubsection{Community Development Process}
\label{sec:community-development-process}

BOOST is developed through the \href{https://www.w3.org/community/boost-01/}{BOOST W3C Community Group} with collaborative input from industry stakeholders, regulatory agencies, and technical experts. The standard implements a \TRU-centric model supporting media-interruption-free tracking, multi-species composition management, and comprehensive plant part categorization across 33 interconnected entities.

\begin{informative}[title=Working Group Leadership]
\begin{itemize}
    \item \textbf{Chair:} Peter Tittmann (Carbon Direct)
    \item \textbf{Technical Contributors:} Industry partners, certification bodies, and regulatory agencies  
    \item \textbf{Community Participants:} 15+ active members from across the biomass supply chain
\end{itemize}
\end{informative}

\subsubsection{Current Development Status}
\label{sec:development-status}

\begin{important}[title=Current Version Information]
\textbf{Current Version:} {{VERSION}} - Complete BOOST Documentation Build System with integrated HTML and PDF generation

\textbf{Recent Enhancements:}
\begin{itemize}
    \item Consolidated documentation architecture with ERD Navigator integration
    \item Complete Resources \& Community section with presentations and meetings
    \item Enhanced entity cross-references and interactive navigation  
    \item Migrated all ReSpec content to unified Bikeshed system while preserving ERD functionality
    \item Interactive ERD Navigator with 33 entities across 7 thematic areas
\end{itemize}
\end{important}

\subsubsection{Participation and Feedback}
\label{sec:participation}

\begin{informative}[title=How to Contribute]
\begin{itemize}
    \item \textbf{GitHub Repository:} \url{https://github.com/carbondirect/BOOST}
    \item \textbf{Issues and Feedback:} Submit via GitHub Issues for technical discussions
    \item \textbf{Community Group:} Join the \href{https://www.w3.org/community/boost-01/}{BOOST W3C Community Group}
    \item \textbf{Interactive Tools:} Use the ERD Navigator to explore and provide schema feedback
\end{itemize}

\textbf{Meeting Schedule:} Regular working group meetings with notes and action items published via GitHub
\end{informative}