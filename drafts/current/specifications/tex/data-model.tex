% Data Model Architecture Section
% Converted from includes/data-model.inc.md

The BOOST data model provides a comprehensive framework for representing all aspects of biomass supply chain operations. The model consists of 33 interconnected entities that work together to provide complete traceability from forest to final product.

\subsection{Key Features}
\label{sec:data-model-features}

\subsubsection{Comprehensive Entity System}
\label{sec:comprehensive-entity-system}

\begin{important}[title=Complete Data Model Coverage]
\begin{itemize}
    \item \textbf{33 Interconnected Entities} - Complete data model covering all aspects of biomass supply chains across 7 thematic areas
    \item \textbf{JSON-LD Validation} - Structured schemas with business rules and examples  
    \item \textbf{Interactive ERD Navigator} - Dynamic exploration with GitHub discussion integration
    \item \textbf{Sustainability Claims} - Species-specific claims with inheritance through processing
\end{itemize}
\end{important}

\subsubsection{Enhanced Geographic Integration}
\label{sec:enhanced-geographic-integration}

\begin{informative}[title=Spatial Data Management]
\begin{itemize}
    \item \textbf{GeoJSON Compliance} - Spatial data support for all location-aware entities
    \item \textbf{California Agency Ready} - Administrative boundary and jurisdiction tracking
    \item \textbf{Supply Base Management} - Infrastructure mapping with harvest sites and transportation routes
\end{itemize}
\end{informative}

The data model implements a hub-and-spoke architecture with \TRU{} as the central hub. All other entities \MUST{} maintain direct or indirect relationships to TRUs to ensure complete traceability.

\subsection{Entity Organization by Thematic Areas}
\label{sec:entity-thematic-areas}

The 33 BOOST entities are organized into 7 thematic areas (see \tabref{tab:entity-thematic-areas}):

\begin{table}[H]
\centering
\caption{BOOST Entity Organization by Thematic Areas}
\label{tab:entity-thematic-areas}
\begin{tabular}{@{}llr@{}}
\toprule
\textbf{Thematic Area} & \textbf{Description} & \textbf{Count} \\
\midrule
\coretraceability{Core Traceability} & Central tracking infrastructure & 5 \\
\organizational{Organizational Foundation} & Business entities and certifications & 6 \\
\materialsupply{Material \& Supply Chain} & Material definitions and supply management & 7 \\
\transaction{Transaction Management} & Business transaction processing & 3 \\
\sustainability{Measurement \& Verification} & Measurement records and claims & 4 \\
\geographic{Geographic \& Tracking} & Spatial data and location services & 2 \\
\reporting{Compliance \& Reporting} & Analytics, reporting, and regulatory compliance & 6 \\
\midrule
\textbf{Total} & & \textbf{33} \\
\bottomrule
\end{tabular}
\end{table}

\begin{normative}[title=Entity Relationship Requirements]
All entities \MUST{} follow the hub-and-spoke design pattern with direct or indirect relationships to \TRU{} entities to maintain complete traceability chain integrity.
\end{normative}

\subsection{Foreign Key Conventions}
\label{sec:foreign-key-conventions}

All foreign key relationships \MUST{} follow the EntityNameId pattern:
\begin{itemize}
    \item Field names \MUST{} end with ``Id''
    \item Field names \MUST{} reference the target entity name in PascalCase
    \item Examples: \field{OrganizationId}, \field{TraceableUnitId}, \field{GeographicDataId}
\end{itemize}

\begin{normative}[title=Foreign Key Naming Convention]
Implementations \MUST{} validate that all foreign key field names follow the EntityNameId pattern to ensure consistent referential integrity across the data model.
\end{normative}