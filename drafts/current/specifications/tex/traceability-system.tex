% BOOST Traceability System Section
% Converted from includes/traceability-system.inc.md

The BOOST Traceability System implements a comprehensive approach to biomass supply chain tracking that eliminates the traditional weak points where traceability is lost during material transfers and processing operations.

\subsection{Key Implementation Features}
\label{sec:key-implementation-features}

\subsubsection{Continuous Traceability Framework}
\label{sec:traceability-continuous-framework}

\begin{important}[title=Comprehensive Traceability Approach]
\TRU{} entities maintain continuous identification through biometric signatures and optical pattern recognition, eliminating dependency on physical tags or attachments that can be lost or damaged during handling and processing operations.
\end{important}

\subsubsection{Critical Tracking Points}
\label{sec:critical-tracking-points}

The system establishes standardized measurement and verification infrastructure with flexible configurations. The standard three-point configuration includes:

\begin{itemize}
    \item \textbf{\enum{harvest\_site}} - Initial TRU creation with biometric capture and volume measurement
    \item \textbf{\enum{consolidation\_point}} - Transportation consolidation points with reconciliation validation (formerly \enum{skid\_road}/\enum{forest\_road})
    \item \textbf{\enum{mill\_entrance}} - Processing facility entry points with final verification before transformation
\end{itemize}

BOOST supports 7 tracking point types total (\enum{harvest\_site}, \enum{consolidation\_point}, \enum{mill\_entrance}, \enum{transfer\_station}, \enum{storage\_facility}, \enum{quality\_control\_point}, \enum{mobile\_processing\_unit}) enabling flexible configurations from 2-point minimum to 5+ point extended setups based on operational complexity.

\begin{normative}[title=Critical Tracking Point Requirements]
Implementations \MUST{} support measurement and verification at all three critical tracking points to ensure complete traceability chain integrity.
\end{normative}

\subsubsection{Multi-Species Support}
\label{sec:multi-species-support}

Species-specific tracking capabilities enable:

\begin{itemize}
    \item Individual species identification within mixed material flows
    \item Species-specific sustainability claim application and inheritance
    \item Detailed composition tracking with percentage validation
    \item Regulatory compliance for jurisdiction-specific species requirements
\end{itemize}

\begin{informative}[title=Species Composition Validation]
\coretraceability{SpeciesComponent} entities provide detailed composition tracking with automatic percentage validation to ensure accuracy in multi-species materials.
\end{informative}

\subsubsection{Complete Processing Chain Documentation}
\label{sec:processing-chain-documentation}

\coretraceability{MaterialProcessing} entities provide comprehensive audit trails by:

\begin{itemize}
    \item Linking input TRUs to output TRUs for every transformation
    \item Tracking plant part changes and transformations during processing
    \item Validating volume and mass conservation across processing steps
    \item Supporting split and merge operations with complete genealogy tracking
\end{itemize}

\begin{normative}[title=Processing Chain Requirements]
All material transformations \MUST{} be documented through \entity{MaterialProcessing} entities that maintain complete input-to-output traceability with validated volume and mass conservation.
\end{normative}