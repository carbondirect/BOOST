\documentclass{article}

% ================================
% BOOST SPECIFICATION DOCUMENT
% ================================
% Main LaTeX document for BOOST W3C Community Group Specification
% Utilizes boost-spec.sty package for professional W3C formatting
% Generated from schema-driven specification build system

\usepackage{boost-spec}
\usepackage[utf8]{inputenc}
% \usepackage{microtype}  % Commented out - causes font expansion issues in CI
\usepackage{float}
\usepackage{graphicx}
\usepackage{caption}
\usepackage{subcaption}

% ================================
% DOCUMENT METADATA
% ================================

\title{Biomass Open Origin Standard for Tracking (BOOST)}
\author{Peter Tittmann, Chair -- BOOST Working Group, Senior Scientist -- Carbon Direct \and Liam Killroy, Carbon Direct \and BOOST W3C Community Group}
\date{August 2025}

% W3C Specification Metadata - handled by \makew3ctitle
% \w3cstatus{W3C Community Group Final Specification}
% \w3cversion{2.8.0}
% \w3cdate{August 6, 2025}
% \w3ceditors{Peter Tittmann, Carbon Direct}
% \w3ccontributors{See Acknowledgments section}
% \w3crepository{https://github.com/carbondirect/BOOST}

% ================================
% DOCUMENT SETUP
% ================================

% Hyperref configuration for PDF bookmarks and links
\hypersetup{
    pdftitle={Biomass Open Origin Standard for Tracking (BOOST)},
    pdfauthor={BOOST W3C Community Group},
    pdfsubject={Biomass Open Origin Standard for Tracking},
    pdfkeywords={biomass, traceability, W3C, data standard, supply chain, LCFS, certification},
    pdfcreator={LaTeX with boost-spec.sty},
    pdfproducer={BOOST Community Group},
    colorlinks=true,
    linkcolor=w3cblue,
    citecolor=reference,
    urlcolor=w3cblue,
    bookmarksopen=true,
    bookmarksopenlevel=2
}

% Page layout and headers (Latin Modern fonts matching minted documentation)
\pagestyle{fancy}
\fancyhf{}
\fancyhead[L]{\small BOOST Specification}
\fancyhead[R]{\small {{VERSION}}}
\fancyfoot[C]{\small\thepage}
\renewcommand{\headrulewidth}{0.4pt}
\renewcommand{\footrulewidth}{0pt}
\setlength{\headheight}{15pt}

% Table of contents formatting
\setcounter{tocdepth}{3}
\setcounter{secnumdepth}{4}

\begin{document}

% ================================
% TITLE PAGE
% ================================

\boosttitle

% ================================
% FRONT MATTER
% ================================

\newpage
\pagenumbering{roman}

% Abstract
\begin{abstract}
The Biomass Open Origin Standard for Tracking (BOOST) data standard defines a comprehensive, interoperable framework for tracking biomass materials through complex supply chains. BOOST enables transparent, verifiable, and consistent data exchange to support sustainability verification, regulatory compliance, and supply chain integrity across the biomass economy. The standard implements a TraceableUnit (TRU)-centric model supporting continuous traceability, multi-species composition management, and comprehensive plant part categorization across 33 interconnected entities organized into 7 thematic areas.
\end{abstract}

% Status of This Document
\begin{w3cstatus}
This specification was published by the Biomass Open Origin Standard for Tracking (BOOST) W3C Community Group (\url{https://www.w3.org/community/boost-01/}). It is not a W3C Standard nor is it on the W3C Standards Track. Please note that under the W3C Community Final Specification Agreement (FSA) other conditions apply. Learn more about W3C Community and Business Groups at \url{https://www.w3.org/community/}.

This document is governed by the W3C Community License Agreement (CLA). A human-readable summary is available at \url{https://www.w3.org/community/about/process/cla-deed/}.

Publication as a Community Group Report does not imply endorsement by the W3C Membership. This is a draft document and may be updated, replaced or obsoleted by other documents at any time. It is inappropriate to cite this document as other than work in progress.
\end{w3cstatus}

% How to Give Feedback
\section*{How to Give Feedback}
This specification is primarily developed on GitHub (\url{https://github.com/carbondirect/BOOST}). The best way to contribute to this specification is to:

\begin{enumerate}
    \item File issues and suggestions in the BOOST GitHub repository (\url{https://github.com/carbondirect/BOOST/issues})
    \item Submit pull requests for specific changes
    \item Participate in community discussions via GitHub Discussions (\url{https://github.com/carbondirect/BOOST/discussions})
    \item Join the W3C Community Group mailing list (\url{https://lists.w3.org/Archives/Public/public-boost-01/}) for broader discussions
\end{enumerate}

% Table of Contents
\newpage
\tableofcontents

% List of Tables
\newpage
\listoftables

% List of Figures  
\newpage
\listoffigures

% ================================
% MAIN CONTENT
% ================================

\newpage
\pagenumbering{arabic}

% ================================
% PART I: FOUNDATION (Context & Requirements)
% ================================

% Executive Summary
\section{Executive Summary}
\label{sec:executive-summary}
% AUTO-GENERATED - DO NOT EDIT
% Generated from: config/narrative_sources/executive_summary.yaml
% To modify this content, edit the source file and regenerate

\subsection{What BOOST Accomplishes}
\label{sec:boost-purpose}

The Biomass Open Origin Standard for Tracking (BOOST) solves the fundamental challenge of maintaining continuous, verifiable traceability through biomass supply chains—from standing trees to processed biofuels—without losing data integrity at critical transfer points where materials change hands, locations, or physical states.

\textbf{Core Problem Addressed}: Traditional biomass tracking systems break down during material transfers, processing operations, and aggregation points, creating traceability gaps where data continuity is lost or becomes unreliable. This forces businesses to maintain separate, incompatible tracking systems for different regulatory requirements while increasing documentation burden and compliance costs.

\textbf{BOOST's Solution}: A unified data standard built around \textbf{TraceableUnits (TRUs)} that maintain their identity throughout the supply chain using progressive identification methods, technology-appropriate data capture, and comprehensive field structures that simultaneously satisfy multiple regulatory frameworks.


\subsection{Overarching Design Principles}
\label{sec:design-principles}

\subsubsection{\textbf{TraceableUnits (TRUs) as Foundation}}

The TraceableUnits (TRUs) entity is the central organizing concept of the entire BOOST standard. Every other entity in the 33-entity data model either creates, modifies, measures, or references TraceableUnits (TRUs)s. This TRU-centric approach reflects the physical reality of biomass operations:

\begin{itemize}
\item \textbf{Harvest Level}: Individual logs, piles, or volume aggregations become initial TRUs with appropriate identification methods
\item \textbf{Transport Level}: TRUs move through critical tracking points (harvest\_site $\rightarrow$ skid\_road $\rightarrow$ forest\_road $\rightarrow$ mill\_entrance) while maintaining identity
\item \textbf{Processing Level}: Input TRUs are transformed into output TRUs with complete genealogical linkage
\item \textbf{Compliance Level}: TRU data aggregates automatically generate required documentation for multiple regulatory programs
\end{itemize}
\subsubsection{\textbf{Physical Reality Mapping}}

BOOST's structure directly mirrors real-world biomass supply chain operations:

1. \textbf{Harvest Operations}: TRUs are created when biomass is harvested, with multi-method identification capturing unique characteristics through technology-appropriate methods
2. \textbf{Transportation Flow}: Critical tracking points represent actual physical locations where custody changes hands or materials are aggregated
3. \textbf{Processing Stages}: Material transformations (whole logs $\rightarrow$ chips $\rightarrow$ pellets $\rightarrow$ biofuels) create new TRUs with documented input-output relationships
4. \textbf{Multi-Species Reality}: Mixed-species piles and processing batches are represented through SpeciesComponent entities that maintain individual species data within composite TRUs


\subsection{Continuous Traceability Framework}
\label{sec:continuous-traceability}

Unlike conventional systems that lose traceability when materials are transferred between different tracking systems, BOOST maintains continuous data linkage through:

\begin{itemize}
\item \textbf{Identity Persistence}: Progressive identification methods that maintain continuity through physical handling and transportation
\item \textbf{Relationship Preservation}: Parent-child TRU relationships that track material splits, merges, and transformations
\item \textbf{Automated Reconciliation}: Volume conservation validation and measurement reconciliation across all tracking points
\end{itemize}

\subsection{Value Proposition: Why BOOST Reduces Costs and Complexity}
\label{sec:value-proposition}

\subsubsection{\textbf{multi-method identification Benefits}}

BOOST's progressive identification methods provides significant operational advantages over conventional single-method tracking approaches:

\textbf{Traditional Approach Limitations}:
\begin{itemize}
\item Manual tagging systems require physical labels that can be lost, damaged, or mislabeled
\item Operator data entry creates human error points and increases labor costs
\item Separate tracking for different species/grades multiplies documentation burden
\item System incompatibilities force duplicate data entry for different regulatory requirements
\end{itemize}
\textbf{BOOST multi-method identification Advantages}:
\begin{itemize}
\item \textbf{Technology-Appropriate Deployment}: Progressive identification methods from RFID tags to optical scanning based on operational readiness levels
\item \textbf{Method Redundancy}: Primary and secondary identification methods ensure continuity when individual methods fail
\item \textbf{Scalable Implementation}: Start with proven methods (RFID, QR codes) and evolve to advanced methods (optical scanning) as technology matures
\item \textbf{Cost-Effective Transition}: Incremental technology adoption minimizes upfront investment while maximizing long-term capability
\end{itemize}
\textbf{Technology Readiness Approach}: BOOST supports identification methods across all Technology Readiness Levels (TRL 1-9), enabling immediate deployment with proven methods while maintaining upgrade paths to emerging technologies as they mature for production use.

\subsubsection{\textbf{Regulatory Compliance Through Strategic Field Combinations}}

BOOST's field structure enables rapid compliance documentation for multiple regulatory frameworks by strategically combining standard data elements:

\textbf{California Low Carbon Fuel Standard (LCFS) Compliance}:
\begin{itemize}
\item Pathway CI calculation: \texttt{LCFSPathway.pathwayCI} + \texttt{EnergyCarbon.benchmarkCI}
\item Volume reporting: \texttt{TraceableUnit.totalVolumeM3} + \texttt{MaterialProcessing.outputQuantity}
\item Feedstock verification: \texttt{SupplyBase.supplyBaseType} + \texttt{GeographicData} location validation
\end{itemize}
\textbf{Renewable Fuel Standard (RFS) Compliance}:
\begin{itemize}
\item Renewable identification: \texttt{Material.materialCategory} + \texttt{SpeciesComponent.species}
\item Volume tracking: \texttt{Transaction.quantityM3} aggregated across supply chain stages
\item Quality documentation: \texttt{TraceableUnit.qualityGrade} + \texttt{MoistureContent} measurements
\end{itemize}
\textbf{EU RED-II Compliance}:
\begin{itemize}
\item Sustainability certification: \texttt{CertificationScheme.schemeType} + \texttt{Certificate} validity
\item Land use verification: \texttt{GeographicData} + \texttt{SupplyBase.supplyBaseType}
\item Greenhouse gas calculations: \texttt{EnergyCarbon.ghgEmissions} + processing stage data
\end{itemize}
\textbf{Multi-Framework Efficiency}: A single BOOST implementation captures all required data fields, eliminating the need for separate tracking systems and enabling automatic generation of compliance documentation for all major regulatory programs.


\subsection{Tolerance Standards and Practical Considerations}
\label{sec:tolerance-standards}

BOOST recognizes that real-world biomass supply chains cannot achieve perfect volume/mass conservation or species composition accuracy:

\textbf{Volume Tolerance Standards}:
\begin{itemize}
\item \textbf{CARB Standard}: ±0.5% volume tolerance for LCFS reporting (as documented on specification page 57)
\item \textbf{Processing Tolerances}: Pelletizing operations may have higher acceptable variance due to moisture content changes and material densification
\item \textbf{De Minimis Thresholds}: Material losses during transport, handling, and processing below 1% can be treated as operational variance rather than tracking errors
\end{itemize}
\textbf{Numeric Precision Guidelines}:
\begin{itemize}
\item \textbf{Carbon Intensity Values}: 2 decimal places (e.g., benchmarkCI: 94.17) for regulatory reporting precision
\item \textbf{Volume Measurements}: 3 decimal places for cubic meter quantities to maintain accuracy across aggregation operations
\item \textbf{Composition Percentages}: 1 decimal place for species composition within multi-species TRUs
\end{itemize}
\textbf{Field Inclusion Rationale}: BOOST includes sufficient data complexity to meet regulatory requirements while maintaining practical implementability. Fields are included based on:
\begin{itemize}
\item Direct regulatory requirement across one or more major programs (LCFS, RFS, EU-RED)
\item Operational necessity for maintaining traceability integrity
\item Industry standard practice for biomass chain of custody documentation
\end{itemize}
This strategic field selection ensures that BOOST implementations serve real-world business purposes by reducing compliance documentation burden while maintaining the data integrity required for sustainability verification and regulatory reporting across multiple jurisdictions.




% Chapter 1: Introduction
\section{Introduction}
\label{sec:introduction}
% Introduction Section for BOOST Specification
% Converted from includes/introduction.inc.md

The Biomass Open-Source Traceability (BOOST) data standard defines a comprehensive, interoperable framework for tracking biomass materials through complex supply chains. BOOST enables transparent, verifiable, and consistent data exchange to support sustainability verification, regulatory compliance, and supply chain integrity across the biomass economy.

\subsubsection{Community Development Process}
\label{sec:community-development-process}

BOOST is developed through the \href{https://www.w3.org/community/boost-01/}{BOOST W3C Community Group} with collaborative input from industry stakeholders, regulatory agencies, and technical experts. The standard implements a \TRU-centric model supporting media-interruption-free tracking, multi-species composition management, and comprehensive plant part categorization across 33 interconnected entities.

\begin{informative}[title=Working Group Leadership]
\begin{itemize}
    \item \textbf{Chair:} Peter Tittmann (Carbon Direct)
    \item \textbf{Technical Contributors:} Industry partners, certification bodies, and regulatory agencies  
    \item \textbf{Community Participants:} 15+ active members from across the biomass supply chain
\end{itemize}
\end{informative}

\subsubsection{Current Development Status}
\label{sec:development-status}

\begin{important}[title=Current Version Information]
\textbf{Current Version:} v2.8.0 - Integrated ERD Navigator + Bikeshed Documentation System

\textbf{Recent Enhancements:}
\begin{itemize}
    \item Consolidated documentation architecture with ERD Navigator integration
    \item Complete Resources \& Community section with presentations and meetings
    \item Enhanced entity cross-references and interactive navigation  
    \item Migrated all ReSpec content to unified Bikeshed system while preserving ERD functionality
    \item Interactive ERD Navigator with 33 entities across 7 thematic areas
\end{itemize}
\end{important}

\subsubsection{Participation and Feedback}
\label{sec:participation}

\begin{informative}[title=How to Contribute]
\begin{itemize}
    \item \textbf{GitHub Repository:} \url{https://github.com/carbondirect/BOOST}
    \item \textbf{Issues and Feedback:} Submit via GitHub Issues for technical discussions
    \item \textbf{Community Group:} Join the \href{https://www.w3.org/community/boost-01/}{BOOST W3C Community Group}
    \item \textbf{Interactive Tools:} Use the ERD Navigator to explore and provide schema feedback
\end{itemize}

\textbf{Meeting Schedule:} Regular working group meetings with notes and action items published via GitHub
\end{informative}

\subsection{Purpose and Scope}
\label{sec:purpose-scope}

This specification defines the BOOST (Biomass Open Origin Standard for Tracking) data standard for biomass supply chain tracking and verification. The standard provides:

\begin{itemize}
    \item A unified data model for biomass custody transfers
    \item Format constraints for serializing chain of custody data
    \item Integration specifications for certification systems
    \item Regulatory compliance frameworks for multiple jurisdictions
\end{itemize}

\subsection{Background and Motivation}
\label{sec:background-motivation}
% Background and Motivation Section for BOOST Specification
% Converted from includes/background.inc.md

The development of comprehensive biomass traceability systems addresses critical needs for sustainability verification, regulatory compliance, and supply chain transparency in the growing biomass economy. This standard enables interoperability between reporting systems, registries, and certification bodies.

\begin{informative}[title=Funding and Jurisdictional Context]
The initial version of this data standard is funded through a grant from the California Department of Conservation, with an initial focus on California as the jurisdictional context while maintaining broad applicability to generalized biomass chain of custody requirements.
\end{informative}

\subsection{Relationship to Existing Standards}
\label{sec:existing-standards}

BOOST builds upon and integrates with established standards including:
\begin{itemize}
    \item ISO 38200~\cite{ISO38200} Chain of custody of wood and wood-based products
    \item SBP Standard 4~\cite{SBP-STANDARD-4} and SBP Standard 5~\cite{SBP-STANDARD-5} from Sustainable Biomass Partnership
    \item FSC-STD-40-004~\cite{FSC-STD-40-004} Forest Stewardship Council certification standards
    \item PEFC-ST-2002~\cite{PEFC-ST-2002} Programme for Endorsement of Forest Certification standards
    \item California LCFS~\cite{CA-LCFS} Low Carbon Fuel Standard requirements
    \item EU RED II~\cite{EU-RED-II} European Union Renewable Energy Directive II
\end{itemize}

\subsection{Community Group Process}
\label{sec:community-process}
% Community Group Process Section for BOOST Specification
% Converted from includes/community-process.inc.md

This specification was developed through the W3C Community Group process with balanced stakeholder participation including civil society organizations, government agencies, small and large businesses, and independent technical experts. Recruitment and engagement efforts were made to avoid overrepresentation of any single stakeholder group.

\begin{normative}[title=W3C Community Group Process]
The group operates under the Community and Business Group Process, seeking organizational licensing commitments under the W3C Community Contributor License Agreement (CLA) for all substantive contributions.
\end{normative}

% Chapter 2: Use Cases and Requirements
\section{Use Cases and Requirements}
\label{sec:use-cases}
% Use Cases and Requirements Section
% Placeholder

\subsection{Primary Use Cases}
\label{sec:primary-use-cases}

BOOST addresses the following primary use cases:

\subsubsection{California Biomass Supply Chain Tracking}
\label{sec:california-use-case}

\begin{itemize}
    \item Forest management organization harvests certified timber
    \item Processing facilities transform raw materials into biofuels
    \item Transportation companies maintain chain of custody
    \item Regulatory agencies verify compliance with LCFS requirements
\end{itemize}

\subsubsection{Multi-Certification Scheme Management}
\label{sec:multi-cert-use-case}

\begin{itemize}
    \item Single TRU maintains multiple certification claims (FSC, SBP, PEFC)
    \item Processing operations preserve claim integrity
    \item Species-specific claims apply to mixed-species materials
    \item Third-party verification validates claim accuracy
\end{itemize}

% Chapter 3: Conformance
\section{Conformance}
\label{sec:conformance}
% Conformance Section for BOOST Specification  
% Converted from includes/conformance.inc.md

This section describes the conformance requirements for BOOST implementations. The key words "\MUST", "\MUSTNOT", "\REQUIRED", "\SHALL", "\SHALLNOT", "\SHOULD", "\SHOULDNOT", "\RECOMMENDED", "\NOTRECOMMENDED", "\MAY", and "\OPTIONAL" in this document are to be interpreted as described in RFC~2119~\cite{RFC2119} and RFC~8174~\cite{RFC8174} when, and only when, they appear in all capitals, as shown here.

\begin{normative}[title=RFC 2119 Keyword Interpretation]
All RFC 2119 keywords in this specification are formatted using the standard conventions and carry the normative meanings defined in RFC 2119 and RFC 8174.
\end{normative}

% ================================
% PART II: CORE SPECIFICATION
% ================================

% Chapter 4: BOOST Traceability System
\section{BOOST Traceability System}
\label{sec:traceability-system}
% BOOST Traceability System Section
% Converted from includes/traceability-system.inc.md

The BOOST Traceability System implements a comprehensive approach to biomass supply chain tracking that eliminates the traditional weak points where traceability is lost during material transfers and processing operations.

\subsection{Key Implementation Features}
\label{sec:key-implementation-features}

\subsubsection{Continuous Traceability Framework}
\label{sec:traceability-continuous-framework}

\begin{important}[title=Comprehensive Traceability Approach]
\TRU{} entities maintain continuous identification through biometric signatures and optical pattern recognition, eliminating dependency on physical tags or attachments that can be lost or damaged during handling and processing operations.
\end{important}

\subsubsection{Three Critical Tracking Points}
\label{sec:three-critical-tracking-points}

The system establishes standardized measurement and verification infrastructure at:

\begin{itemize}
    \item \textbf{\enum{harvest\_site}} - Initial TRU creation with biometric capture and volume measurement
    \item \textbf{\enum{skid\_road}/\enum{forest\_road}} - Transportation consolidation points with reconciliation validation
    \item \textbf{\enum{mill\_entrance}} - Processing facility entry points with final verification before transformation
\end{itemize}

\begin{normative}[title=Critical Tracking Point Requirements]
Implementations \MUST{} support measurement and verification at all three critical tracking points to ensure complete traceability chain integrity.
\end{normative}

\subsubsection{Multi-Species Support}
\label{sec:multi-species-support}

Species-specific tracking capabilities enable:

\begin{itemize}
    \item Individual species identification within mixed material flows
    \item Species-specific sustainability claim application and inheritance
    \item Detailed composition tracking with percentage validation
    \item Regulatory compliance for jurisdiction-specific species requirements
\end{itemize}

\begin{informative}[title=Species Composition Validation]
\coretraceability{SpeciesComponent} entities provide detailed composition tracking with automatic percentage validation to ensure accuracy in multi-species materials.
\end{informative}

\subsubsection{Complete Processing Chain Documentation}
\label{sec:processing-chain-documentation}

\coretraceability{MaterialProcessing} entities provide comprehensive audit trails by:

\begin{itemize}
    \item Linking input TRUs to output TRUs for every transformation
    \item Tracking plant part changes and transformations during processing
    \item Validating volume and mass conservation across processing steps
    \item Supporting split and merge operations with complete genealogy tracking
\end{itemize}

\begin{normative}[title=Processing Chain Requirements]
All material transformations \MUST{} be documented through \entity{MaterialProcessing} entities that maintain complete input-to-output traceability with validated volume and mass conservation.
\end{normative}

% Chapter 5: Data Model Architecture
\section{Data Model Architecture}
\label{sec:data-model}
% Data Model Architecture Section
% Converted from includes/data-model.inc.md

The BOOST data model provides a comprehensive framework for representing all aspects of biomass supply chain operations. The model consists of 33 interconnected entities that work together to provide complete traceability from forest to final product.

\subsection{Key Features}
\label{sec:data-model-features}

\subsubsection{Comprehensive Entity System}
\label{sec:comprehensive-entity-system}

\begin{important}[title=Complete Data Model Coverage]
\begin{itemize}
    \item \textbf{33 Interconnected Entities} - Complete data model covering all aspects of biomass supply chains across 7 thematic areas
    \item \textbf{JSON-LD Validation} - Structured schemas with business rules and examples  
    \item \textbf{Interactive ERD Navigator} - Dynamic exploration with GitHub discussion integration
    \item \textbf{Sustainability Claims} - Species-specific claims with inheritance through processing
\end{itemize}
\end{important}

\subsubsection{Enhanced Geographic Integration}
\label{sec:enhanced-geographic-integration}

\begin{informative}[title=Spatial Data Management]
\begin{itemize}
    \item \textbf{GeoJSON Compliance} - Spatial data support for all location-aware entities
    \item \textbf{California Agency Ready} - Administrative boundary and jurisdiction tracking
    \item \textbf{Supply Base Management} - Infrastructure mapping with harvest sites and transportation routes
\end{itemize}
\end{informative}

The data model implements a hub-and-spoke architecture with \TRU{} as the central hub. All other entities \MUST{} maintain direct or indirect relationships to TRUs to ensure complete traceability.

\subsection{Entity Organization by Thematic Areas}
\label{sec:entity-thematic-areas}

The 33 BOOST entities are organized into 7 thematic areas (see \tabref{tab:entity-thematic-areas}):

\begin{table}[H]
\centering
\caption{BOOST Entity Organization by Thematic Areas}
\label{tab:entity-thematic-areas}
\begin{tabular}{@{}llr@{}}
\toprule
\textbf{Thematic Area} & \textbf{Description} & \textbf{Count} \\
\midrule
\coretraceability{Core Traceability} & Central tracking infrastructure & 5 \\
\organizational{Organizational Foundation} & Business entities and certifications & 6 \\
\materialsupply{Material \& Supply Chain} & Material definitions and supply management & 7 \\
\transaction{Transaction Management} & Business transaction processing & 3 \\
\sustainability{Measurement \& Verification} & Measurement records and claims & 4 \\
\geographic{Geographic \& Tracking} & Spatial data and location services & 2 \\
\reporting{Compliance \& Reporting} & Analytics, reporting, and regulatory compliance & 6 \\
\midrule
\textbf{Total} & & \textbf{33} \\
\bottomrule
\end{tabular}
\end{table}

\begin{normative}[title=Entity Relationship Requirements]
All entities \MUST{} follow the hub-and-spoke design pattern with direct or indirect relationships to \TRU{} entities to maintain complete traceability chain integrity.
\end{normative}

\subsection{Foreign Key Conventions}
\label{sec:foreign-key-conventions}

All foreign key relationships \MUST{} follow the EntityNameId pattern:
\begin{itemize}
    \item Field names \MUST{} end with ``Id''
    \item Field names \MUST{} reference the target entity name in PascalCase
    \item Examples: \field{OrganizationId}, \field{TraceableUnitId}, \field{GeographicDataId}
\end{itemize}

\begin{normative}[title=Foreign Key Naming Convention]
Implementations \MUST{} validate that all foreign key field names follow the EntityNameId pattern to ensure consistent referential integrity across the data model.
\end{normative}

% Chapter 6: Complete Entity Definitions
\section{Core Data Entities}
\label{sec:complete-entities}

% All 7 thematic areas with comprehensive entity tables
% AUTO-GENERATED - DO NOT EDIT
% Generated from: thematic area configuration and entity schemas
% To modify this content, edit the source files and regenerate
% Core Traceability Entities


\subsubsection{Traceable Unit}
\label{sec:entity-traceable-unit}

Unique biomass tracking unit with BOOST traceability system integration

\begin{informative}[title=Entity Relationships]
This entity references the following entities:
\begin{itemize}
    \item \field{traceableUnitId} → \entity{Traceable Unit} (Unique ID for each TRU)
    \item \field{operatorId} → \entity{Operator} (Foreign key to operator)
\end{itemize}
\end{informative}

% Traceable Unit Entity Table
% Auto-generated from JSON schema

\begin{entitytable}{Traceable Unit}
\textbf{\field{createdTimestamp}} & string (date-time) & When the TRU was created \\
\textbf{\field{harvestGeographicDataId}} & string (pattern) & Harvest location - uses EntityNameId convention referencing GeographicData \\
\textbf{\field{harvesterId}} & string (pattern) & Foreign key to harvesting organization \\
\textbf{\field{isMultiSpecies}} & boolean & True if contains multiple species \\
\textbf{\field{materialTypeId}} & string (pattern) & Foreign key to Material entity (reference table) \\
\textbf{\field{totalVolumeM3}} & number (≥0) & Total volume of the traceable unit in cubic meters \\
\textbf{\field{traceableUnitId}} & string (pattern) & Unique ID for each TRU \\
\textbf{\field{uniqueIdentifier}} & string & Biometric signature, RFID tag, or QR code \\
\textbf{\field{unitType}} & enum(4 values) & Type of traceable unit \\
\field{assortmentType} & enum(4 values) & Type of wood assortment \\
\field{attachedInformation} & array<string> & All data linked to this TRU \\
\field{childTraceableUnitIds} & array<string> & For split/merge operations (Phase 2) \\
\field{currentGeographicDataId} & string (pattern) & Current location - uses EntityNameId convention referencing GeographicData \\
\field{currentStatus} & enum(4 values) & Current status of the TRU (Phase 2) \\
\field{lastUpdated} & string (date-time) & Timestamp of the most recent data update \\
\field{mediaBreakFlags} & array<string> & Points where data continuity was lost (Phase 2) \\
\field{operatorId} & string (pattern) & Foreign key to operator \\
\field{parentTraceableUnitId} & string & For split/merge operations (Phase 2) \\
\field{processingHistory} & array<string> & Complete processing chain references (Phase 2) \\
\field{qualityGrade} & enum(5 values) & Quality grade classification \\
\field{sustainabilityCertification} & string & FSC, PEFC, etc. claims (Phase 2) \\
\end{entitytable}



\subsubsection{Material Processing}
\label{sec:entity-material-processing}

Processing operations that transform TRUs with plant part tracking

\begin{informative}[title=Entity Relationships]
This entity references the following entities:
\begin{itemize}
    \item \field{operatorId} → \entity{Operator} ()
\end{itemize}
\end{informative}

% AUTO-GENERATED - DO NOT EDIT
% Generated from: material_processing/validation_schema.json and material_processing_dictionary.md
% To modify this content, edit the source files and regenerate
% Material Processing Entity Table

\begin{entitytable}{Material Processing}
\textbf{\field{@context} & object (structured) & No description provided \\
\textbf{\field{@id} & string (uri) & No description provided \\
\textbf{\field{@type} & enum(MaterialProcessing) & No description provided \\
\textbf{\field{inputTraceableUnitId} & string (pattern) & No description provided \\
\textbf{\field{inputVolume} & number (≥0) & No description provided \\
\textbf{\field{outputTraceableUnitId} & string (pattern) & No description provided \\
\textbf{\field{outputVolume} & number (≥0) & No description provided \\
\textbf{\field{processTimestamp} & string (date-time) & No description provided \\
\textbf{\field{processType} & enum(6 values) & No description provided \\
\textbf{\field{processingId} & string (pattern) & No description provided \\
\field{equipmentUsed} & string & No description provided \\
\field{inputComposition} & string & No description provided \\
\field{inputPlantParts} & object (structured) & Plant parts in input TRU before processing \\
\field{operatorId} & string (pattern) & No description provided \\
\field{outputComposition} & string & No description provided \\
\field{outputPlantParts} & object (structured) & Plant parts in output TRU after processing \\
\field{plantPartLosses} & object (structured) & Volume losses by plant part during processing \\
\field{plantPartTransformations} & array<object> & Specific plant part transformations during processing \\
\field{processingGeographicDataId} & string (pattern) & No description provided \\
\field{qualityMetrics} & string & No description provided \\
\field{volumeLoss} & number (≥0) & No description provided \\
\end{entitytable}



\subsubsection{Processing History}
\label{sec:entity-processing-history}

Complete timeline of processing events with moisture tracking

\begin{informative}[title=Entity Relationships]
This entity references the following entities:
\begin{itemize}
    \item \field{processingHistoryId} → \entity{Processing History} (Unique identifier for the processing history record)
    \item \field{traceableUnitId} → \entity{Traceable Unit} (Foreign key to TRU this history record belongs to)
    \item \field{materialProcessingId} → \entity{Material Processing} (Foreign key to MaterialProcessing operation)
    \item \field{operatorId} → \entity{Operator} (Foreign key to operator who performed processing)
\end{itemize}
\end{informative}

% Processing History Entity Table
% Auto-generated from JSON schema

\begin{entitytable}{Processing History}
\textbf{\field{inputTRUIds}} & array<string> & Array of input TRU IDs (multiple for merge operations) \\
\textbf{\field{materialProcessingId}} & string & Foreign key to MaterialProcessing operation \\
\textbf{\field{outputTRUIds}} & array<string> & Array of output TRU IDs (multiple for split operations) \\
\textbf{\field{processSequenceNumber}} & integer & Sequential order of this processing step for the TRU \\
\textbf{\field{processingEventType}} & enum(7 values) & Type of processing event \\
\textbf{\field{processingHistoryId}} & string & Unique identifier for the processing history record \\
\textbf{\field{timestamp}} & string (date-time) & When this processing step occurred \\
\textbf{\field{traceableUnitId}} & string & Foreign key to TRU this history record belongs to \\
\field{claimInheritanceData} & object (structured) & Sustainability claim inheritance tracking \\
\field{equipmentUsed} & string & Equipment used for this processing step \\
\field{isCurrentProcessingState} & boolean & True if this represents the current processing state \\
\field{mediaBreakData} & object (structured) & Media break detection and recovery information \\
\field{nextProcessingHistoryIds} & array<string> & Array of next processing history record IDs (for split operations) \\
\field{operatorId} & string & Foreign key to operator who performed processing \\
\field{plantPartTransformation} & string & Summary of plant part changes during processing \\
\field{previousProcessingHistoryId} & ['string', 'null'] & Foreign key to previous processing history record (forms chain) \\
\field{processingDuration} & string (pattern) & ISO 8601 duration format for processing time \\
\field{processingGeographicDataId} & string & Foreign key to location where processing occurred \\
\field{qualityChangeDescription} & string & Description of quality changes during processing \\
\field{speciesCompositionChange} & enum(5 values) & How species composition changed during processing \\
\field{volumeChangeRatio} & number (≥0, ≤2.0) & Ratio of output volume to input volume (1.0 = no change) \\
\field{volumeConservationData} & object (structured) & Volume conservation validation data \\
\end{entitytable}



\subsubsection{Location History}
\label{sec:entity-location-history}

Historical movement records of TRUs

\begin{informative}[title=Entity Relationships]
This entity references the following entities:
\begin{itemize}
    \item \field{locationHistoryId} → \entity{Location History} ()
    \item \field{traceableUnitId} → \entity{Traceable Unit} ()
    \item \field{geographicDataId} → \entity{Geographic Data} ()
    \item \field{materialProcessingId} → \entity{Material Processing} ()
    \item \field{operatorId} → \entity{Operator} ()
\end{itemize}
\end{informative}

% Location History Entity Table
% Auto-generated from JSON schema

\begin{entitytable}{Location History}
\textbf{\field{geographicDataId}} & string (pattern) & Foreign key to geographic location data \\
\textbf{\field{isCurrentLocation}} & boolean & Boolean flag indicating iscurrentlocation status \\
\textbf{\field{locationEventType}} & enum(5 values) & Enumerated value for locationeventtype \\
\textbf{\field{locationHistoryId}} & string (pattern) & Unique identifier for the locationhistory \\
\textbf{\field{timestamp}} & string (date-time) & Timestamp field value \\
\textbf{\field{traceableUnitId}} & string (pattern) & Foreign key to the traceable unit \\
\field{distanceTraveled} & number (≥0) & Distancetraveled field value \\
\field{equipmentUsed} & string & Equipmentused field value \\
\field{lastUpdated} & string (date-time) & Timestamp of the most recent data update \\
\field{materialProcessingId} & ['string', 'null'] & Unique identifier for the materialprocessing \\
\field{notes} & string & Notes field value \\
\field{operatorId} & string (pattern) & Foreign key to the operator who performed this action \\
\field{transportMethod} & enum(5 values) & Enumerated value for transportmethod \\
\field{verificationMethods} & array<string> & Array of verificationmethods values \\
\end{entitytable}



\subsubsection{Biometric Identifier}
\label{sec:entity-biometric-identifier}

BiometricIdentifier entity in BOOST data model

\begin{informative}[title=Entity Relationships]
This entity references the following entities:
\begin{itemize}
    \item \field{traceableUnitId} → \entity{Traceable Unit} (Foreign key to TraceableUnit entity)
    \item \field{trackingPointId} → \entity{Tracking Point} ()
\end{itemize}
\end{informative}

% Biometric Identifier Entity Table
% Auto-generated from JSON schema

\begin{entitytable}{Biometric Identifier}
\textbf{\field{biometricId}} & string (pattern) & No description provided \\
\textbf{\field{biometricSignature}} & string & No description provided \\
\textbf{\field{captureMethod}} & enum(optical_scanner, photo_analysis) & No description provided \\
\textbf{\field{captureTimestamp}} & string (date-time) & No description provided \\
\textbf{\field{traceableUnitId}} & string & No description provided \\
\field{captureGeographicDataId} & string & No description provided \\
\field{speciesBiometrics} & array<string> & No description provided \\
\field{trackingPointId} & string & No description provided \\
\end{entitytable}



% Organizational Foundation Entities
% Auto-generated from JSON schemas


\subsection{Organization}
\label{sec:entity-organization}

Organization entity with geographic data references and certification management capabilities for Phase 2 BOOST traceability system enhancements

\begin{informative}[title=Entity Relationships]
This entity references the following entities:
\begin{itemize}
    \item \field{organizationId} → \entity{Organization} (Unique identifier for the organization)
\end{itemize}
\end{informative}

**[View Organization in ERD Navigator](erd-navigator/index.html?focus=Organization)**

% Organization Entity Table
% Auto-generated from JSON schema

\begin{entitytable}{Organization}
\textbf{\field{organizationId}} & string (pattern) & Unique identifier for the organization \\
\textbf{\field{organizationName}} & string & Legal name of the organization \\
\textbf{\field{organizationType}} & enum(10 values) & Type of organization \\
\field{airDistrictPermit} & string & Air quality management district permit identifier \\
\field{bioramContractId} & string (pattern) & BioRAM competitive procurement contract identifier (external system reference) \\
\field{bioramEligibilityStatus} & enum(4 values) & Current BioRAM program eligibility status \\
\field{bioramFacilityId} & string (pattern) & BioRAM facility identifier for program tracking (external system reference) \\
\field{bioramRegistrationId} & string (pattern) & CEC BioRAM registration identifier for biomass power facilities (external sys... \\
\field{calFireJurisdiction} & string & CAL FIRE unit or jurisdiction for facility area \\
\field{californiaSRA} & boolean & Whether facility operates within California State Responsibility Area \\
\field{certifications} & array<string> & List of certification IDs held by organization \\
\field{contactEmail} & string (email) & Primary contact email address \\
\field{contactPhone} & string (pattern) & Primary contact phone number \\
\field{establishedDate} & string (date) & Date organization was established \\
\field{facilityCapacity} & object (structured) & Facility production or handling capacity for LCFS reporting \\
\field{fireHazardZoneDesignation} & enum(4 values) & CAL FIRE fire hazard severity zone designation for facility location \\
\field{gridInterconnectionPoint} & string & Grid interconnection substation or transmission point \\
\field{lastUpdated} & string (date-time) & Timestamp of the most recent data update \\
\field{lcfsRegistrationId} & string (pattern) & CARB LCFS registration identifier for regulated entities (external system ref... \\
\field{operationalAreas} & array<string> & List of geographic areas where organization operates \\
\field{operationalStatus} & enum(4 values) & Current operational status of the organization \\
\field{powerPurchaseAgreementId} & string & Power purchase agreement identifier with utility offtaker \\
\field{primaryGeographicDataId} & string (pattern) & Foreign key to primary operational location \\
\field{regulatedEntityType} & enum(5 values) & LCFS regulated entity classification \\
\field{taxId} & string & Tax identification number \\
\field{utilityOfftaker} & string & Utility company purchasing power under BioRAM contract \\
\field{website} & string (uri) & Organization website URL \\
\end{entitytable}



\subsection{Certificate}
\label{sec:entity-certificate}

Certificate entity representing formal certification records issued by certification bodies

\begin{informative}[title=Entity Relationships]
This entity references the following entities:
\begin{itemize}
    \item \field{certificateId} → \entity{Certificate} (Standard certificate identifier using CERT- pattern)
    \item \field{OrganizationId} → \entity{Organization} (Uses EntityNameId convention referencing Organization receiving the certificate)
\end{itemize}
\end{informative}

**[View Certificate in ERD Navigator](erd-navigator/index.html?focus=Certificate)**

% Certificate Entity Table
% Auto-generated from JSON schema

\begin{entitytable}{Certificate}
\textbf{\field{CertificationBodyId}} & string (pattern) & Uses EntityNameId convention referencing CertificationBody \\
\textbf{\field{CertificationSchemeId}} & string (pattern) & Uses EntityNameId convention referencing CertificationScheme \\
\textbf{\field{OrganizationId}} & string (pattern) & Uses EntityNameId convention referencing Organization receiving the certificate \\
\textbf{\field{certificateId}} & string (pattern) & Standard certificate identifier using CERT- pattern \\
\textbf{\field{certificateNumber}} & string (pattern) & Official certificate number (primary key) \\
\textbf{\field{dateOfExpiry}} & string (date) & Certificate expiry date \\
\textbf{\field{dateOfIssue}} & string (date) & Date of certificate issuance \\
\textbf{\field{scopeOfCertification}} & string & Summary of certification coverage \\
\textbf{\field{status}} & enum(4 values) & Current certificate status \\
\textbf{\field{versionNumber}} & string & Version identifier of the certification standard \\
\field{VerificationStatementId} & string (pattern) & Uses EntityNameId convention referencing VerificationStatement for third-part... \\
\field{auditSchedule} & object (structured) & Scheduled audit information \\
\field{certificateDocument} & string (uri) & Link or reference to certificate document \\
\field{conditionalRequirements} & array<object> & Special conditions or requirements \\
\field{suspensionHistory} & array<object> & History of certificate suspensions \\
\field{versionYear} & integer & Year of the standard's relevant version release \\
\end{entitytable}



\subsection{CertificationBody}
\label{sec:entity-certification-body}

Certification Body entity representing independent organizations authorized to issue certificates

**[View CertificationBody in ERD Navigator](erd-navigator/index.html?focus=CertificationBody)**

% CertificationBody Entity Table
% Auto-generated from JSON schema

\begin{entitytable}{CertificationBody}
\textbf{\field{accreditationStatus}} & enum(5 values) & Current accreditation status \\
\textbf{\field{authorizedSchemes}} & array<string> & List of certification schemes the CB can certify under \\
\textbf{\field{cbId}} & string (pattern) & Unique identifier for the certification body (primary key) \\
\textbf{\field{cbName}} & string & Official name of the certification body \\
\textbf{\field{cbType}} & enum(4 values) & Type or category of certification body \\
\textbf{\field{contactInformation}} & object (structured) & Contact details for the certification body \\
\textbf{\field{operationalRegions}} & array<string> & Geographic regions where CB operates (ISO country codes) \\
\textbf{\field{validityPeriod}} & object (structured) & Period of CB authorization \\
\field{accreditationBody} & string & Organization that accredited this CB \\
\field{performanceMetrics} & object (structured) & CB performance and quality indicators \\
\field{specializations} & array<string> & Specific areas of certification expertise \\
\end{entitytable}



\subsection{CertificationScheme}
\label{sec:entity-certification-scheme}

CertificationScheme entity defining certification standards and requirements with geographic applicability for Phase 2 BOOST traceability system enhancements

**[View CertificationScheme in ERD Navigator](erd-navigator/index.html?focus=CertificationScheme)**

% CertificationScheme Entity Table
% Auto-generated from JSON schema

\begin{entitytable}{CertificationScheme}
\textbf{\field{certificationSchemeId}} & string (pattern) & Unique identifier for the certification scheme \\
\textbf{\field{issuingOrganizationId}} & string (pattern) & Foreign key to organization that issues this scheme \\
\textbf{\field{schemeName}} & string & Official name of the certification scheme \\
\textbf{\field{schemeType}} & enum(6 values) & Type of certification scheme \\
\field{applicableGeographicAreas} & array<string> & Geographic areas where this scheme is applicable \\
\field{auditRequirements} & string & Audit and verification requirements \\
\field{chainOfCustodyRequirements} & string & Chain of custody tracking and documentation requirements \\
\field{claimTypes} & array<string> & Types of claims supported by this scheme \\
\field{complianceTolerances} & object (structured) & Only regulatory compliance tolerances - equipment and process tolerances are ... \\
\field{documentationRequirements} & array<string> & Required documentation and record-keeping \\
\field{eligibleMaterialTypes} & array<string> & Material types eligible for this certification scheme \\
\field{lastUpdated} & string (date-time) & Timestamp of the most recent data update \\
\field{schemeDescription} & string & Detailed description of the certification scheme \\
\field{schemeStandard} & string & Standard or version identifier \\
\field{validityPeriod} & string & Typical validity period for certifications under this scheme \\
\field{website} & string (uri) & Official website for the certification scheme \\
\end{entitytable}



\subsection{Audit}
\label{sec:entity-audit}

Audit entity in BOOST data model

\begin{informative}[title=Entity Relationships]
This entity references the following entities:
\begin{itemize}
    \item \field{auditId} → \entity{Audit} ()
    \item \field{organizationId} → \entity{Organization} ()
\end{itemize}
\end{informative}

**[View Audit in ERD Navigator](erd-navigator/index.html?focus=Audit)**

% Audit Entity Table
% Auto-generated from JSON schema

\begin{entitytable}{Audit}
\textbf{\field{auditDate}} & string (date) & Auditdate field value \\
\textbf{\field{auditId}} & string (pattern) & Unique identifier for the audit \\
\textbf{\field{auditType}} & enum(Initial, Surveillance, Transfer) & Enumerated value for audittype \\
\textbf{\field{organizationId}} & string (pattern) & Foreign key to the associated organization \\
\field{auditGeographicDataId} & string (pattern) & Unique identifier for the auditgeographicdata \\
\field{cbId} & string (pattern) & Unique identifier for the cb \\
\field{findings} & string & Findings field value \\
\field{reportUrl} & string (uri) & Reporturl field value \\
\end{entitytable}



\subsection{BOOST Operator Entity Validation Schema}
\label{sec:entity-operator}

Validation schema for personnel and operator management within the BOOST biomass chain of custody system

\begin{informative}[title=Entity Relationships]
This entity references the following entities:
\begin{itemize}
    \item \field{operatorId} → \entity{Operator} (Unique identifier for the operator (Primary Key))
    \item \field{organizationId} → \entity{Organization} (Employing organization - uses EntityNameId convention referencing Organization)
\end{itemize}
\end{informative}

**[View BOOST Operator Entity Validation Schema in ERD Navigator](erd-navigator/index.html?focus=BOOSTOperatorEntityValidationSchema)**

% BOOST Operator Entity Validation Schema Entity Table
% Auto-generated from JSON schema

\begin{entitytable}{BOOST Operator Entity Validation Schema}
\textbf{\field{hireDate}} & string (date) & Date when operator started employment \\
\textbf{\field{isActive}} & boolean & Current employment status - true if actively employed \\
\textbf{\field{lastUpdated}} & string (date-time) & Timestamp of last record modification \\
\textbf{\field{operatorId}} & string (pattern) & Unique identifier for the operator (Primary Key) \\
\textbf{\field{operatorName}} & string & Full name of the operator \\
\textbf{\field{operatorType}} & enum(10 values) & Type/role of operator within the supply chain \\
\textbf{\field{organizationId}} & string (pattern) & Employing organization - uses EntityNameId convention referencing Organization \\
\field{certifications} & array<string> & Array of certifications held by the operator \\
\field{contactInfo} & ['string', 'null'] & Phone/email contact information \\
\field{employeeId} & ['string', 'null'] & Internal employee identification number \\
\field{equipmentAuthorizations} & array<string> & Equipment the operator is authorized to operate \\
\field{skillsQualifications} & array<string> & Relevant skills and qualifications \\
\field{supervisorOperatorId} & ['string', 'null'] & Foreign key reference to direct supervisor operator (optional) \\
\end{entitytable}



% Material & Supply Chain Entities
% Auto-generated from JSON schemas


\subsubsection{Material}
\label{sec:entity-material}

Material types and specifications

** [View Material in ERD Navigator](erd-navigator/index.html?focus=Material)**

% Material Entity Table
% Auto-generated from JSON schema

\begin{entitytable}{Material}
\textbf{\field{materialCategory}} & enum(softwood, hardwood, mixed) & No description provided \\
\textbf{\field{materialName}} & string & No description provided \\
\textbf{\field{materialTypeId}} & string & No description provided \\
\field{applicablePlantParts} & array<string> & Plant parts included in this material type \\
\field{applicableProcessingTypes} & array<string> & No description provided \\
\field{carbonStorageRate} & string & No description provided \\
\field{defaultAssortmentTypes} & string & No description provided \\
\field{density} & string & No description provided \\
\field{energyContent} & string & No description provided \\
\field{excludedPlantParts} & array<string> & Plant parts excluded from this material type \\
\field{lastUpdated} & string (date-time) & No description provided \\
\field{plantPartProcessingSpecs} & object & Processing specifications by plant part \\
\field{standardMoistureContent} & string & No description provided \\
\field{standardQualityGrades} & string & No description provided \\
\field{typicalSpecies} & array<string> & No description provided \\
\end{entitytable}



\subsubsection{Species Component}
\label{sec:entity-species-component}

Species composition within TRUs

** [View Species Component in ERD Navigator](erd-navigator/index.html?focus=SpeciesComponent)**

% Species Component Entity Table
% Auto-generated from JSON schema

\begin{entitytable}{Species Component}
\textbf{\field{componentId}} & string & Unique identifier for the species component \\
\textbf{\field{percentageByVolume}} & number ($\geq$ 0, $\leq$ 100) & Percentage of total TRU volume for this species \\
\textbf{\field{species}} & string & Species name (common or scientific) \\
\textbf{\field{traceableUnitId}} & string & Foreign key back reference to TraceableUnit \\
\textbf{\field{volumeM3}} & number ($\geq$ 0) & Volume of this species within the TRU in cubic meters \\
\field{ageYears} & integer & Estimated age in years \\
\field{carbonStorage} & string & CO2 data for this species component \\
\field{dbhCm} & number ($\geq$ 0) & Diameter at breast height in centimeters \\
\field{defects} & array<string> & List of defects or quality issues \\
\field{harvestTimestamp} & string (date-time) & When this species was harvested \\
\field{harvestingMethod} & enum(4 values) & Method used to harvest this species \\
\field{heightM} & number ($\geq$ 0) & Average tree height in meters \\
\field{lastUpdated} & string (date-time) & Timestamp of the most recent data update \\
\field{moistureContent} & number ($\geq$ 0, $\leq$ 100) & Moisture content as percentage \\
\field{plantPartComposition} & object & Plant part breakdown within this species component \\
\field{primaryPlantPart} & enum(17 values) & Primary plant part represented by this species component \\
\field{qualityGrade} & string & Species-specific quality grade \\
\field{scientificName} & string & Scientific/Latin name of the species \\
\field{sourceGeographicDataId} & string & Foreign key to geographic origin of this species \\
\field{structuralClassification} & enum(5 values) & Functional classification of the primary plant part \\
\end{entitytable}



\subsubsection{Supplier}
\label{sec:entity-supplier}

Supplier entity in BOOST data model

\begin{informative}[title=Entity Relationships]
This entity references the following entities:
\begin{itemize}
    \item \field{supplierId} → \entity{Supplier} ()
\end{itemize}
\end{informative}

** [View Supplier in ERD Navigator](erd-navigator/index.html?focus=Supplier)**

% Supplier Entity Table
% Auto-generated from JSON schema

\begin{entitytable}{Supplier}
\textbf{\field{supplierId}} & string & No description provided \\
\textbf{\field{supplierName}} & string & No description provided \\
\field{GeographicDataId} & string & Supplier location - uses EntityNameId convention referencing GeographicData \\
\field{address} & string & No description provided \\
\field{certificateCode} & string & No description provided \\
\field{claim} & string & No description provided \\
\field{supplierType} & string & No description provided \\
\end{entitytable}



\subsubsection{Customer}
\label{sec:entity-customer}

Customer entity in BOOST data model

\begin{informative}[title=Entity Relationships]
This entity references the following entities:
\begin{itemize}
    \item \field{customerId} → \entity{Customer} (Unique identifier for the customer)
\end{itemize}
\end{informative}

** [View Customer in ERD Navigator](erd-navigator/index.html?focus=Customer)**

% Customer Entity Table
% Auto-generated from JSON schema

\begin{entitytable}{Customer}
\textbf{\field{customerId}} & string (pattern) & Unique identifier for the customer \\
\textbf{\field{customerName}} & string & No description provided \\
\field{GeographicDataId} & string (pattern) & Customer location - uses EntityNameId convention referencing GeographicData \\
\field{address} & string & No description provided \\
\end{entitytable}



\subsubsection{Supply Base}
\label{sec:entity-supply-base}

SupplyBase entity in BOOST data model

\begin{informative}[title=Entity Relationships]
This entity references the following entities:
\begin{itemize}
    \item \field{OrganizationId} → \entity{Organization} (Managing organization - uses EntityNameId convention referencing Organization)
\end{itemize}
\end{informative}

** [View Supply Base in ERD Navigator](erd-navigator/index.html?focus=SupplyBase)**

% Supply Base Entity Table
% Auto-generated from JSON schema

\begin{entitytable}{Supply Base}
\textbf{\field{OrganizationId}} & string & Managing organization - uses EntityNameId convention referencing Organization \\
\textbf{\field{description}} & string & No description provided \\
\textbf{\field{supplyBaseId}} & string & No description provided \\
\textbf{\field{supplyBaseName}} & string & No description provided \\
\field{GeographicDataId} & string & Supply base location - uses EntityNameId convention referencing GeographicData \\
\field{equipmentDeployment} & array<string> & No description provided \\
\field{forestRoads} & array<string> & No description provided \\
\field{harvestSites} & array<string> & No description provided \\
\field{skidRoads} & array<string> & No description provided \\
\field{speciesAvailable} & array<string> & No description provided \\
\field{traceableUnitIds} & array<string> & No description provided \\
\end{entitytable}



\subsubsection{Supply Base Report}
\label{sec:entity-supply-base-report}

SupplyBaseReport entity in BOOST data model

\begin{informative}[title=Entity Relationships]
This entity references the following entities:
\begin{itemize}
    \item \field{organizationId} → \entity{Organization} ()
\end{itemize}
\end{informative}

** [View Supply Base Report in ERD Navigator](erd-navigator/index.html?focus=SupplyBaseReport)**

% Supply Base Report Entity Table
% Auto-generated from JSON schema

\begin{entitytable}{Supply Base Report}
\textbf{\field{organizationId}} & string & No description provided \\
\textbf{\field{preparationDate}} & string (date) & No description provided \\
\textbf{\field{sbrId}} & string & No description provided \\
\field{publicationUrl} & string (uri) & No description provided \\
\field{reportGeographicDataId} & string & No description provided \\
\field{sourcingPractices} & string & No description provided \\
\field{supplyBaseIds} & array<string> & Array of SupplyBase IDs that this report covers \\
\field{supplyBaseSummary} & string & No description provided \\
\field{sustainabilityMeasures} & string & No description provided \\
\end{entitytable}



\subsubsection{Equipment}
\label{sec:entity-equipment}

Equipment entity representing forestry machinery and equipment used in biomass harvesting and processing operations

\begin{informative}[title=Entity Relationships]
This entity references the following entities:
\begin{itemize}
    \item \field{equipmentId} → \entity{Equipment} (Unique identifier for the equipment)
    \item \field{organizationId} → \entity{Organization} (Foreign key to owning organization)
\end{itemize}
\end{informative}

** [View Equipment in ERD Navigator](erd-navigator/index.html?focus=Equipment)**

% Equipment Entity Table
% Auto-generated from JSON schema

\begin{entitytable}{Equipment}
\textbf{\field{equipmentId}} & string (pattern) & Unique identifier for the equipment \\
\textbf{\field{equipmentName}} & string & Descriptive name of the equipment \\
\textbf{\field{equipmentType}} & enum(12 values) & Type of forestry equipment \\
\textbf{\field{operationalStatus}} & enum(5 values) & Current operational status of the equipment \\
\textbf{\field{organizationId}} & string (pattern) & Foreign key to owning organization \\
\field{acquisitionCost} & number ($\geq$ 0) & Equipment acquisition cost in USD \\
\field{acquisitionDate} & string (date) & Date equipment was acquired by organization \\
\field{assignedTrackingPointId} & string (pattern) & Foreign key to current location/assignment \\
\field{certifications} & array<string> & Equipment certifications (safety, emissions, etc.) \\
\field{currentOperatorId} & string (pattern) & Foreign key to current operator (if assigned) \\
\field{insuranceInfo} & object (structured) & Equipment insurance information \\
\field{lastUpdated} & string (date-time) & Timestamp of the most recent data update \\
\field{maintenanceSchedule} & object (structured) & Maintenance schedule information \\
\field{manufacturer} & string & Equipment manufacturer \\
\field{model} & string & Equipment model designation \\
\field{notes} & string & Additional notes or comments about the equipment \\
\field{serialNumber} & string & Manufacturer serial number \\
\field{specifications} & object (structured) & Technical specifications for the equipment \\
\field{yearManufactured} & integer & Year the equipment was manufactured \\
\end{entitytable}



% Transaction Management Entities
% Auto-generated from JSON schemas


\subsubsection{Transaction}
\label{sec:entity-transaction}

Transaction entity in BOOST data model

\begin{informative}[title=Entity Relationships]
This entity references the following entities:
\begin{itemize}
    \item \field{transactionId} → \entity{Transaction} (Unique identifier for the business transaction)
    \item \field{OrganizationId} → \entity{Organization} (Primary organization involved in transaction (seller/supplier))
    \item \field{CustomerId} → \entity{Customer} (Customer organization (buyer) - uses EntityNameId convention referencing Customer entity)
\end{itemize}
\end{informative}

**🗂️ [View Transaction in ERD Navigator](erd-navigator/index.html?focus=Transaction)**

% Transaction Entity Table
% Auto-generated from JSON schema

\begin{entitytable}{Transaction}
\textbf{\field{CustomerId}} & string (pattern) & Customer organization (buyer) - uses EntityNameId convention referencing Cust... \\
\textbf{\field{OrganizationId}} & string (pattern) & Primary organization involved in transaction (seller/supplier) \\
\textbf{\field{contractCurrency}} & enum(9 values) & Currency code for contract value \\
\textbf{\field{contractValue}} & number (≥0, ≤999999999.99) & Total monetary value of the transaction \\
\textbf{\field{transactionDate}} & string (date) & Date of business agreement \\
\textbf{\field{transactionId}} & string (pattern) & Unique identifier for the business transaction \\
\textbf{\field{transactionStatus}} & enum(6 values) & Current status of business transaction \\
\field{BrokerOrganizationId} & ['string', 'null'] & Optional intermediary broker organization - uses EntityNameId convention refe... \\
\field{GeographicDataId} & string (pattern) & Primary transaction location - uses EntityNameId convention referencing Geogr... \\
\field{LcfsPathwayId} & string (pattern) & CARB-certified pathway identifier for LCFS compliance - uses EntityNameId con... \\
\field{SalesDeliveryDocumentId} & string (pattern) & Foreign key to sales/delivery documentation - uses EntityNameId convention re... \\
\field{complianceRequirements} & array<string> & Regulatory compliance requirements for transaction \\
\field{contractSignedDate} & ['string', 'null'] & Date when contract was executed \\
\field{contractTerms} & enum(8 values) & Incoterms delivery conditions \\
\field{expectedDeliveryDate} & ['string', 'null'] & Expected completion/delivery date \\
\field{financialTerms} & object (structured) & Detailed financial terms and conditions \\
\field{fuelCategory} & enum(10 values) & Category of fuel for LCFS classification \\
\field{fuelVolume} & number (≥0) & Volume of fuel in transaction for LCFS reporting \\
\field{fuelVolumeUnit} & enum(gallons, liters, GGE) & Unit of measurement for fuel volume \\
\field{lastUpdated} & string (date-time) & Timestamp of last modification \\
\field{manipulationTimestamps} & array<string> & Processing step timestamps \\
\field{mediaBreaksDetected} & array<boolean> & Continuity flags per TRU \\
\field{paymentTerms} & string & Payment conditions and timeline \\
\field{reconciliationStatus} & enum(pending, resolved, disputed) & Transaction reconciliation status \\
\field{regulatedPartyRole} & enum(5 values) & Role of regulated party in LCFS transaction \\
\field{reportingPeriod} & string (pattern) & LCFS reporting quarter in YYYY-QN format \\
\field{riskManagement} & object (structured) & Risk management and mitigation terms \\
\field{speciesCompositionAtTransaction} & array<object> & Species breakdown at transaction time \\
\field{traceableUnitIds} & array<string> & TRUs included in this transaction \\
\field{trackingPointIds} & array<string> & Location trail references \\
\end{entitytable}



\subsubsection{Transaction Batch}
\label{sec:entity-transaction-batch}

TransactionBatch entity in BOOST data model

\begin{informative}[title=Entity Relationships]
This entity references the following entities:
\begin{itemize}
    \item \field{transactionId} → \entity{Transaction} (Foreign key to parent business transaction)
    \item \field{claimId} → \entity{Claim} (Foreign key to primary sustainability claim)
\end{itemize}
\end{informative}

**🗂️ [View Transaction Batch in ERD Navigator](erd-navigator/index.html?focus=TransactionBatch)**

% Transaction Batch Entity Table
% Auto-generated from JSON schema

\begin{entitytable}{Transaction Batch}
\textbf{\field{batchStatus}} & enum(6 values) & Current status of the physical batch \\
\textbf{\field{quantity}} & number (≥0) & Physical quantity of material in this batch \\
\textbf{\field{quantityUnit}} & enum(7 values) & Unit of measurement for quantity \\
\textbf{\field{traceableUnitIds}} & array<string> & Array of TRU IDs included in this batch \\
\textbf{\field{transactionBatchId}} & string (pattern) & Unique identifier for the physical material batch \\
\textbf{\field{transactionId}} & string (pattern) & Foreign key to parent business transaction \\
\field{additionalClaimIds} & array<string> & Array of secondary claim IDs \\
\field{batchCreatedDate} & string (date-time) & When the batch was prepared/created \\
\field{certificationValidation} & object (structured) & Certification and compliance validation data \\
\field{claimId} & ['string', 'null'] & Foreign key to primary sustainability claim \\
\field{deliveryDate} & ['string', 'null'] & Actual delivery timestamp \\
\field{deliveryGeographicDataId} & ['string', 'null'] & Foreign key to delivery location \\
\field{lastUpdated} & string (date-time) & Timestamp of last modification \\
\field{measurementRecordIds} & array<string> & Array of measurement record IDs \\
\field{mediaBreakDetected} & boolean & Flag indicating if traceability continuity was broken \\
\field{plantPartComposition} & object & Plant part composition breakdown \\
\field{processingHistoryIds} & array<string> & Array of processing history record IDs \\
\field{productionBatchId} & ['string', 'null'] & Foreign key to source production batch \\
\field{qualityGrade} & enum(9 values) & Overall quality grade for the batch \\
\field{qualityMetrics} & object (structured) & Detailed quality assessment metrics \\
\field{reconciliationStatus} & enum(5 values) & Status of volume/quality reconciliation \\
\field{speciesComposition} & array<object> & Species breakdown with percentages \\
\field{trackingHistory} & string & Complete location trail summary \\
\field{transportationData} & object (structured) & Transportation and logistics information \\
\end{entitytable}



\subsubsection{Sales Delivery Document}
\label{sec:entity-sales-delivery-document}

SalesDeliveryDocument entity in BOOST data model

\begin{informative}[title=Entity Relationships]
This entity references the following entities:
\begin{itemize}
    \item \field{transactionId} → \entity{Transaction} ()
\end{itemize}
\end{informative}

**🗂️ [View Sales Delivery Document in ERD Navigator](erd-navigator/index.html?focus=SalesDeliveryDocument)**

% Sales Delivery Document Entity Table
% Auto-generated from JSON schema

\begin{entitytable}{Sales Delivery Document}
\textbf{\field{buyerName}} & string & No description provided \\
\textbf{\field{dateIssued}} & string (date) & No description provided \\
\textbf{\field{documentId}} & string (pattern) & No description provided \\
\textbf{\field{productDescription}} & string & No description provided \\
\textbf{\field{quantity}} & number & No description provided \\
\textbf{\field{sellerName}} & string & No description provided \\
\field{buyerAddress} & string & No description provided \\
\field{certificateCode} & string & No description provided \\
\field{sellerAddress} & string & No description provided \\
\field{transactionId} & string & No description provided \\
\field{transportReference} & string & No description provided \\
\end{entitytable}



% Measurement & Verification Entities
% Auto-generated from JSON schemas


\subsubsection{Measurement Record}
\label{sec:entity-measurement-record}

Quality measurements and dimensional data

\begin{informative}[title=Entity Relationships]
This entity references the following entities:
\begin{itemize}
    \item \field{operatorId} → \entity{Operator} ()
\end{itemize}
\end{informative}

** [View Measurement Record in ERD Navigator](erd-navigator/index.html?focus=MeasurementRecord)**

% Measurement Record Entity Table
% Auto-generated from JSON schema

\begin{entitytable}{Measurement Record}
\textbf{\field{measurementMethod}} & enum(4 values) & No description provided \\
\textbf{\field{measurementTimestamp}} & string (date-time) & No description provided \\
\textbf{\field{recordId}} & string & No description provided \\
\textbf{\field{traceableUnitId}} & string & No description provided \\
\field{lastUpdated} & string (date-time) & No description provided \\
\field{measuredDiameter} & number (\geq 0) & No description provided \\
\field{measuredLength} & number (\geq 0) & No description provided \\
\field{measuredVolume} & number (\geq 0) & No description provided \\
\field{measurementGeographicDataId} & string & No description provided \\
\field{moistureAccuracy} & number (\geq 0) & Estimated accuracy of moisture measurement (± percentage points) \\
\field{moistureContent} & number (\geq 0, \leq 100) & Moisture content as percentage of weight contributed by water (0-100\textback... \\
\field{moistureMethod} & enum(8 values) & Method used to determine moisture content \\
\field{moistureStandard} & enum(5 values) & Standard procedure followed for moisture measurement \\
\field{operatorId} & string & No description provided \\
\field{speciesMeasurements} & array<string> & No description provided \\
\field{trackingPointId} & string & No description provided \\
\end{entitytable}



\subsubsection{Claim}
\label{sec:entity-claim}

Claim entity in BOOST data model

\begin{informative}[title=Entity Relationships]
This entity references the following entities:
\begin{itemize}
    \item \field{claimId} → \entity{Claim} ()
\end{itemize}
\end{informative}

** [View Claim in ERD Navigator](erd-navigator/index.html?focus=Claim)**

% Claim Entity Table
% Auto-generated from JSON schema

\begin{entitytable}{Claim}
\textbf{\field{TraceableUnitId}} & string & Referenced traceable unit - uses EntityNameId convention referencing Traceabl... \\
\textbf{\field{claimId}} & string & No description provided \\
\textbf{\field{claimType}} & enum(9 values) & No description provided \\
\textbf{\field{statement}} & string & No description provided \\
\textbf{\field{validated}} & boolean & No description provided \\
\field{CertificationSchemeId} & string & Certification scheme - uses EntityNameId convention referencing Certification... \\
\field{applicableSpecies} & array<string> & No description provided \\
\field{claimExpiry} & string (date-time) & No description provided \\
\field{claimPercentage} & number (\geq 0, \leq 100) & No description provided \\
\field{claimScope} & enum(4 values) & No description provided \\
\field{evidenceDocumentId} & string & No description provided \\
\field{inheritedFromTRU} & array<string> & No description provided \\
\field{lastUpdated} & string (date-time) & No description provided \\
\field{validatedBy} & string & No description provided \\
\field{validationDate} & string (date-time) & No description provided \\
\end{entitytable}



\subsubsection{Verification Statement}
\label{sec:entity-verification-statement}

VerificationStatement entity in BOOST data model

** [View Verification Statement in ERD Navigator](erd-navigator/index.html?focus=VerificationStatement)**

% Verification Statement Entity Table
% Auto-generated from JSON schema

\begin{entitytable}{Verification Statement}
\textbf{\field{issuingBody}} & string & No description provided \\
\textbf{\field{statementId}} & string & No description provided \\
\textbf{\field{verificationDate}} & string (date) & No description provided \\
\field{scope} & string & No description provided \\
\field{transactionBatchId} & string & No description provided \\
\end{entitytable}



\subsubsection{BOOST Moisture Content Validation Rules}
\label{sec:entity-moisture-content}

Comprehensive validation rules and business logic for moisture content tracking across the BOOST data standard

** [View BOOST Moisture Content Validation Rules in ERD Navigator](erd-navigator/index.html?focus=BOOSTMoistureContentValidationRules)**

% BOOST Moisture Content Validation Rules Entity Table
% Auto-generated from JSON schema

\begin{entitytable}{BOOST Moisture Content Validation Rules}
\end{entitytable}



% Geographic & Tracking Entities
% Auto-generated from JSON schemas


\subsection{Geographic Data}
\label{sec:entity-geographic-data}

GeographicData entity in BOOST data model

**[View Geographic Data in ERD Navigator](erd-navigator/index.html?focus=GeographicData)**

% Geographic Data Entity Table
% Auto-generated from JSON schema

\begin{entitytable}{Geographic Data}
\textbf{\field{dataType}} & enum(7 values) & Type of geographic data \\
\textbf{\field{description}} & string & Human-readable description of the geographic area \\
\textbf{\field{geoJsonData}} & object (structured) & Valid GeoJSON object (Point, Polygon, LineString, etc.) \\
\textbf{\field{geographicDataId}} & string (pattern) & Unique identifier for the geographic data \\
\field{accessRestrictions} & string & Any access restrictions or special conditions \\
\field{accuracy} & number (≥0) & GPS accuracy in meters \\
\field{administrativeRegion} & string & Administrative region or jurisdiction \\
\field{coordinateSystem} & string & Coordinate reference system (e.g., WGS84, UTM Zone 10N) \\
\field{elevationM} & number & Elevation in meters above sea level \\
\field{lastUpdated} & string (date-time) & Timestamp of the most recent data update \\
\end{entitytable}



\subsection{Tracking Point}
\label{sec:entity-tracking-point}

TrackingPoint entity in BOOST data model

\begin{informative}[title=Entity Relationships]
This entity references the following entities:
\begin{itemize}
    \item \field{operatorId} → \entity{Operator} ()
\end{itemize}
\end{informative}

**[View Tracking Point in ERD Navigator](erd-navigator/index.html?focus=TrackingPoint)**

% Tracking Point Entity Table
% Auto-generated from JSON schema

\begin{entitytable}{Tracking Point}
\textbf{\field{equipmentUsed}} & string & No description provided \\
\textbf{\field{establishedTimestamp}} & string (date-time) & No description provided \\
\textbf{\field{geographicDataId}} & string & No description provided \\
\textbf{\field{pointType}} & enum(4 values) & No description provided \\
\textbf{\field{trackingPointId}} & string (pattern) & Unique identifier for the tracking point \\
\field{operatorId} & string & No description provided \\
\end{entitytable}



% Compliance & Reporting Entities
% Auto-generated from JSON schemas


\subsubsection{LCFS Pathway}
\label{sec:entity-lcfs-pathway}

CARB-certified fuel pathway for LCFS compliance with carbon intensity and regulatory attributes

** [View LCFS Pathway in ERD Navigator](erd-navigator/index.html?focus=LCFSPathway)**

% LCFS Pathway Entity Table
% Auto-generated from JSON schema

\begin{entitytable}{LCFS Pathway}
\textbf{\field{caGreetVersion}} & string (pattern) & CA-GREET model version used for pathway certification \\
\textbf{\field{carbonIntensity}} & number (\geq 0, \leq 200) & Certified carbon intensity in gCO2e/MJ \\
\textbf{\field{certificationDate}} & string (date) & CARB pathway certification date \\
\textbf{\field{energyEconomyRatio}} & number (\geq 0.5, \leq 3.0) & Energy economy ratio multiplier for credit calculation \\
\textbf{\field{facilityLocation}} & string & Production facility location (city, state or geographic region) \\
\textbf{\field{feedstockCategory}} & enum(13 values) & Primary feedstock type for pathway \\
\textbf{\field{fuelProduct}} & enum(8 values) & Final fuel product produced \\
\textbf{\field{pathwayId}} & string (pattern) & CARB-assigned pathway identifier \\
\textbf{\field{pathwayType}} & enum(Lookup_Table, Tier_1, Tier_2) & CARB pathway certification tier \\
\textbf{\field{verificationStatus}} & enum(4 values) & Current CARB verification status \\
\field{expirationDate} & string (date) & Pathway certification expiration date \\
\field{facilityCapacity} & number (\geq 0) & Annual production capacity in gallons \\
\field{geographicScope} & enum(4 values) & Geographic applicability of pathway \\
\field{lastUpdated} & string (date-time) & Timestamp of most recent pathway data update \\
\field{processDescription} & string & Brief description of production process \\
\end{entitytable}



\subsubsection{LCFS Reporting}
\label{sec:entity-lcfs-reporting}

Quarterly LCFS compliance report for regulated entities with credit calculations and submission tracking

** [View LCFS Reporting in ERD Navigator](erd-navigator/index.html?focus=LCFSReporting)**

% LCFS Reporting Entity Table
% Auto-generated from JSON schema

\begin{entitytable}{LCFS Reporting}
\textbf{\field{complianceStatus}} & enum(4 values) & Overall compliance status for the reporting period \\
\textbf{\field{netPosition}} & number & Net credit/deficit position (credits - deficits) \\
\textbf{\field{regulatedEntityId}} & string & Reference to regulated Organization entity \\
\textbf{\field{reportingId}} & string (pattern) & Unique identifier for the quarterly report \\
\textbf{\field{reportingPeriod}} & string (pattern) & Reporting quarter in YYYY-QN format \\
\textbf{\field{totalCreditsGenerated}} & number (\geq 0) & Total LCFS credits generated in the reporting period \\
\textbf{\field{totalDeficitsIncurred}} & number (\geq 0) & Total LCFS deficits incurred in the reporting period \\
\textbf{\field{totalFuelVolume}} & number (\geq 0) & Total fuel volume reported in gallons \\
\field{VerificationStatementId} & string & Uses EntityNameId convention referencing VerificationStatement for third-part... \\
\field{calculationParameters} & object (structured) & Calculation parameters used for credit computation \\
\field{complianceMetrics} & object (structured) & Additional compliance and environmental impact metrics \\
\field{lastUpdated} & string (date-time) & Timestamp of most recent report update \\
\field{pathwaySummary} & array<object> & Summary of activity by LCFS pathway \\
\field{reportingDeadline} & string (date) & CARB deadline for report submission \\
\field{submissionDate} & string (date-time) & Date and time report was submitted to CARB \\
\field{transactionIds} & array<string> & Array of Transaction entity IDs included in this report \\
\field{verificationDate} & string (date-time) & Date of third-party verification completion \\
\field{verificationRequired} & boolean & Whether third-party verification is required for this entity \\
\end{entitytable}



\subsubsection{Product Group}
\label{sec:entity-product-group}

ProductGroup entity in BOOST data model

** [View Product Group in ERD Navigator](erd-navigator/index.html?focus=ProductGroup)**

% Product Group Entity Table
% Auto-generated from JSON schema

\begin{entitytable}{Product Group}
\textbf{\field{description}} & string & No description provided \\
\textbf{\field{productCategory}} & enum(solid_biomass, liquid_biofuel, biogas) & No description provided \\
\textbf{\field{productGroupId}} & string & No description provided \\
\textbf{\field{productGroupName}} & string & No description provided \\
\field{certificationRequirements} & array<string> & No description provided \\
\field{classification} & string & No description provided \\
\field{lastUpdated} & string (date-time) & No description provided \\
\field{qualityStandards} & array<string> & No description provided \\
\field{regulatoryClassification} & string & No description provided \\
\field{relatedMaterials} & array<object> & No description provided \\
\field{typicalUses} & array<string> & No description provided \\
\end{entitytable}



\subsubsection{Energy Carbon Data}
\label{sec:entity-energy-carbon-data}

EnergyCarbonData entity in BOOST data model

** [View Energy Carbon Data in ERD Navigator](erd-navigator/index.html?focus=EnergyCarbonData)**

% Energy Carbon Data Entity Table
% Auto-generated from JSON schema

\begin{entitytable}{Energy Carbon Data}
\textbf{\field{dataType}} & enum(7 values) & No description provided \\
\textbf{\field{energyCarbonDataId}} & string & No description provided \\
\textbf{\field{source}} & enum(4 values) & No description provided \\
\textbf{\field{unit}} & enum(8 values) & No description provided \\
\textbf{\field{value}} & number & No description provided \\
\field{caGreetVersion} & string (pattern) & CA-GREET model version used for calculation \\
\field{energyEconomyRatio} & number (\geq 0.5, \leq 3.0) & Energy economy ratio for LCFS credit calculation \\
\field{humidityConditions} & number & No description provided \\
\field{lcfsPathwayType} & enum(4 values) & LCFS pathway tier classification \\
\field{lifeCycleStage} & enum(6 values) & Lifecycle stage for carbon intensity data \\
\field{measurementGeographicDataId} & string & No description provided \\
\field{measurementMethod} & enum(9 values) & No description provided \\
\field{measurementRecordId} & string & No description provided \\
\field{measurementTimestamp} & string (date-time) & No description provided \\
\field{qualityAssurance} & string & No description provided \\
\field{regulatoryBenchmark} & number & CARB regulatory benchmark for comparison (gCO2e/MJ) \\
\field{temperatureConditions} & number & No description provided \\
\field{traceableUnitId} & string & No description provided \\
\end{entitytable}



\subsubsection{Data Reconciliation}
\label{sec:entity-data-reconciliation}

DataReconciliation entity in BOOST data model

\begin{informative}[title=Entity Relationships]
This entity references the following entities:
\begin{itemize}
    \item \field{transactionId} → \entity{Transaction} ()
\end{itemize}
\end{informative}

** [View Data Reconciliation in ERD Navigator](erd-navigator/index.html?focus=DataReconciliation)**

% Data Reconciliation Entity Table
% Auto-generated from JSON schema

\begin{entitytable}{Data Reconciliation}
\textbf{\field{discrepancy}} & number & No description provided \\
\textbf{\field{forestMeasurement}} & number (\geq 0) & No description provided \\
\textbf{\field{millMeasurement}} & number (\geq 0) & No description provided \\
\textbf{\field{reconciliationDate}} & string (date-time) & No description provided \\
\textbf{\field{reconciliationId}} & string & No description provided \\
\textbf{\field{reconciliationStatus}} & enum(pending, resolved, disputed) & No description provided \\
\textbf{\field{traceableUnitId}} & string & No description provided \\
\field{discrepancyReason} & string & No description provided \\
\field{lastUpdated} & string (date-time) & No description provided \\
\field{reconciliationOperator} & string & No description provided \\
\field{resolutionNotes} & string & No description provided \\
\field{speciesDiscrepancies} & array<string> & No description provided \\
\field{tolerancePercentage} & number (\geq 0, \leq 100) & No description provided \\
\field{transactionId} & string & No description provided \\
\end{entitytable}



\subsubsection{Mass Balance Account}
\label{sec:entity-mass-balance-account}

MassBalanceAccount entity in BOOST data model

\begin{informative}[title=Entity Relationships]
This entity references the following entities:
\begin{itemize}
    \item \field{organizationId} → \entity{Organization} ()
\end{itemize}
\end{informative}

** [View Mass Balance Account in ERD Navigator](erd-navigator/index.html?focus=MassBalanceAccount)**

% Mass Balance Account Entity Table
% Auto-generated from JSON schema

\begin{entitytable}{Mass Balance Account}
\textbf{\field{accountId}} & string & No description provided \\
\textbf{\field{currentBalance}} & number & No description provided \\
\textbf{\field{organizationId}} & string & No description provided \\
\textbf{\field{productGroupId}} & string & No description provided \\
\field{balancingPeriod} & string & No description provided \\
\field{conversionFactors} & number & No description provided \\
\field{periodInputs} & number & No description provided \\
\field{periodOutputs} & number & No description provided \\
\end{entitytable}




% Chapter 7: Plant Part Categorization System  
\section{Plant Part Categorization System}
\label{sec:plant-parts}
% Plant Part Categorization System Section

\subsection{Introduction and Regulatory Context}
\label{sec:plant-parts-intro}

The BOOST plant part categorization framework addresses critical regulatory and operational requirements across multiple jurisdictions and applications. This system provides the taxonomic foundation for distinguishing product classifications from physical arrangements, enabling sophisticated supply chain optimization and regulatory compliance.

\subsubsection{Regulatory Drivers}

\textbf{California Department of Food and Agriculture (CDFA) Requirements}
\begin{itemize}
    \item Agricultural biomass classification for food vs. fuel categorization
    \item Privacy protections for farmer data in agricultural residue tracking  
    \item Integration with existing CDFA biomass certification systems
    \item Support for agricultural waste stream optimization programs
\end{itemize}

\textbf{Low Carbon Fuel Standard (LCFS) Compliance}
\begin{itemize}
    \item Feedstock categorization requirements for carbon intensity calculations
    \item Plant part composition tracking for pathway verification
    \item Biogenic carbon accounting across different material components
    \item Alternative fate assessment support for BECCS applications
\end{itemize}

\textbf{Forest Stewardship Council (FSC) Integration}
\begin{itemize}
    \item Chain of custody tracking through plant part transformations
    \item Controlled wood verification for different plant components
    \item Multi-species composition documentation requirements
    \item Value recovery optimization across plant part classifications
\end{itemize}

\subsubsection{Conceptual Framework}

The BOOST system distinguishes between two fundamental attributes:

\textbf{Product Classification vs. Physical Arrangement}
\begin{itemize}
    \item \textbf{Product Classification}: Market destination or intended use (sawlog, pulpwood, biomass, chips)
    \item \textbf{Physical Arrangement}: Spatial organization affecting collection and decomposition (scattered, piled, windrow, stacked)
\end{itemize}

This distinction enables sophisticated LCA and BECCS analysis by capturing both economic intent and operational reality.

\subsection{Standardized Plant Parts Taxonomy}
\label{sec:plant-parts-taxonomy}

Implementations \MUST{} support the following 17 standardized plant parts:

\begin{itemize}
    \item \textbf{\enum{trunk}} - Main stem/bole of tree
    \item \textbf{\enum{heartwood}} - Inner, non-living wood
    \item \textbf{\enum{sapwood}} - Outer, living wood
    \item \textbf{\enum{bark}} - Protective outer layer
    \item \textbf{\enum{branches}} - Secondary stems
    \item \textbf{\enum{leaves}} - Photosynthetic organs
    \item \textbf{\enum{seeds}} - Reproductive structures
    \item \textbf{\enum{roots}} - Below-ground structures
    \item \textbf{\enum{twigs}} - Small branches
    \item \textbf{\enum{cones}} - Seed-bearing structures
    \item \textbf{\enum{needles}} - Coniferous leaves
    \item \textbf{\enum{foliage}} - All leaf matter
    \item \textbf{\enum{crown}} - Above-ground branching structure
    \item \textbf{\enum{stump}} - Remaining base after felling
    \item \textbf{\enum{chips}} - Mechanically processed fragments
    \item \textbf{\enum{sawdust}} - Fine processing residue
    \item \textbf{\enum{pellets}} - Densified processed material
\end{itemize}

\begin{normative}[title=Plant Part Classification Requirements]
All \TRU{} entities \MUST{} specify plant part classification using this standardized taxonomy to ensure consistent categorization across implementations.
\end{normative}

\subsection{Physical Arrangement Framework}
\label{sec:physical-arrangement}

The BOOST system captures spatial organization of biomass materials to support collection planning and LCA analysis. Physical arrangement significantly affects both operational efficiency and environmental impact assessment.

\subsubsection{Arrangement Categories}

\textbf{Scattered Arrangement}
\begin{itemize}
    \item Crowns and branches distributed across forest floor after harvesting
    \item Lower collection efficiency (typically 65-75\%)
    \item Higher decomposition rates due to ground contact and weather exposure
    \item Alternative fate: natural decomposition or wildfire fuel
\end{itemize}

\textbf{Centralized Piles}
\begin{itemize}
    \item Material gathered into specific collection points for efficiency
    \item High collection efficiency (typically 90-95\%)
    \item Moderate decomposition rates depending on pile construction
    \item Optimized for mechanical loading and transport operations
\end{itemize}

\textbf{Windrow Configuration}
\begin{itemize}
    \item Linear arrangements following equipment access patterns
    \item Collection efficiency 80-90\% with mechanical systems
    \item Balanced decomposition rates with partial ground contact
    \item Enables efficient forwarding and chipping operations
\end{itemize}

\textbf{Stacked Arrangements}
\begin{itemize}
    \item Organized vertical stacking for drying and storage
    \item Highest collection efficiency (95\%+) with quality preservation
    \item Lowest decomposition rates when properly ventilated
    \item Premium applications requiring controlled moisture content
\end{itemize}

\subsubsection{LCA and BECCS Integration}

\textbf{Alternative Fate Modeling}
\begin{itemize}
    \item Baseline scenario assessment (decomposition, wildfire, prescribed burning)
    \item Arrangement-specific decomposition rates for carbon accounting
    \item Emissions avoided calculations based on collection vs. baseline
    \item Soil carbon impact assessment from ground contact patterns
\end{itemize}

\textbf{Collection Efficiency Factors}
\begin{itemize}
    \item Energy requirements for different arrangement patterns
    \item Equipment accessibility and operational constraints
    \item Volume recovery rates by arrangement and terrain conditions
    \item Economic optimization through arrangement planning
\end{itemize}

\begin{normative}[title=Physical Arrangement Requirements]
\TRU{} entities \MAY{} include physical arrangement data to support LCA analysis and collection optimization. When included, arrangement data \MUST{} use standardized arrangement types and provide collection efficiency factors.
\end{normative}

% ================================
% PART III: IMPLEMENTATION
% ================================

% Chapter 8: Schema Definitions
\section{Schema Definitions}
\label{sec:schema-definitions}
% Schema Definitions Section
% Placeholder - will be enhanced with schema details

\subsection{JSON Schema Format}
\label{sec:json-schema-format}

All BOOST entity definitions \MUST{} be provided as JSON Schema Draft-07 compliant schemas with the following \REQUIRED{} structure:

\begin{jsonexample}{Entity Schema Structure}
\begin{minted}[fontsize=\small,linenos=false,breaklines=true,tabsize=2]{json}
{
  "schema": {
    "$schema": "http://json-schema.org/draft-07/schema#",
    "$id": "https://github.com/carbondirect/BOOST/schemas/entity-name",
    "title": "Entity Name",
    "type": "object",
    "properties": { },
    "required": [ ]
  }
}
\end{minted}
\end{jsonexample}

\subsection{Business Logic Validation}
\label{sec:business-logic}

Implementations \MUST{} validate entities against 8 categories of business rules:

\begin{enumerate}
    \item \textbf{Volume/Mass Conservation} - Physical conservation laws
    \item \textbf{Temporal Logic} - Date consistency validation
    \item \textbf{Geographic Logic} - Spatial relationship validation
    \item \textbf{Species Composition} - Percentage validation (sum to 100\%)
    \item \textbf{Certification Logic} - Chain of custody validation
    \item \textbf{Regulatory Compliance} - Jurisdiction-specific rules
    \item \textbf{Economic Logic} - Price and payment validation
    \item \textbf{Quality Assurance} - Material quality constraints
\end{enumerate}

\begin{normative}[title=Validation Requirements]
Conforming implementations \MUST{} validate data against BOOST JSON schemas and implement all required business logic validation rules for their conformance level.
\end{normative}

% Chapter 9: Serialization and Exchange
\section{Serialization and Exchange}
\label{sec:serialization}
% Serialization and Exchange Section
% Placeholder

\subsection{JSON-LD as Primary Format}
\label{sec:json-ld}

BOOST data \MUST{} be serializable to JSON-LD format with:
\begin{itemize}
    \item Valid \texttt{@context} referencing BOOST context definition
    \item Entity \texttt{@type} declarations matching schema names  
    \item Unique \texttt{@id} values for all entities
\end{itemize}

\begin{normative}[title=JSON-LD Requirements]
All BOOST data exchanges \MUST{} use valid JSON-LD 1.1 format with appropriate context definitions and semantic annotations.
\end{normative}

% Chapter 10: JSON-LD Context and Semantic Web Integration
\section{JSON-LD Context and Semantic Web Integration}
\label{sec:jsonld-context}
% JSON-LD Context and Semantic Web Integration
% Documentation for BOOST JSON-LD support

\section{JSON-LD Context and Semantic Web Integration}
\label{sec:jsonld-context}

BOOST implements JSON-LD (JSON for Linking Data) as its primary serialization format, enabling semantic web compatibility, data linking, and machine-readable context definitions. This section explains the JSON-LD context structure, semantic annotations, and integration with existing ontologies.

\subsection{JSON-LD Overview}
\label{sec:jsonld-overview}

JSON-LD extends standard JSON with semantic web capabilities through:

\begin{itemize}
    \item \textbf{@context}: Defines mappings between JSON properties and RDF vocabularies
    \item \textbf{@id}: Provides unique identifiers for entities (IRIs)
    \item \textbf{@type}: Specifies the semantic type of an entity
    \item \textbf{@vocab}: Sets a default vocabulary for properties
    \item \textbf{Linked Data}: Enables connections between distributed datasets
\end{itemize}

\subsection{BOOST Context Definition}
\label{sec:boost-context}

The BOOST JSON-LD context maps entity properties to established vocabularies:

\begin{jsonexample}{BOOST Core Context}
{
  "@context": {
    "schema": "http://schema.org/",
    "prov": "http://www.w3.org/ns/prov#",
    "gs1": "https://gs1.org/voc/",
    "biomass": "http://example.org/biomass#",
    "geo": "http://www.w3.org/2003/01/geo/wgs84_pos#",
    "qudt": "http://qudt.org/schema/qudt/",
    "unit": "http://qudt.org/vocab/unit/",
    
    "TraceableUnit": "biomass:TraceableUnit",
    "Organization": "schema:Organization",
    "Transaction": "schema:Order",
    "MaterialProcessing": "prov:Activity",
    
    "traceableUnitId": {
      "@id": "schema:identifier",
      "@type": "schema:Text"
    },
    "organizationId": {
      "@id": "schema:identifier",
      "@type": "schema:Text"
    },
    "createdAt": {
      "@id": "schema:dateCreated",
      "@type": "xsd:dateTime"
    },
    "modifiedAt": {
      "@id": "schema:dateModified",
      "@type": "xsd:dateTime"
    }
  }
}
\end{jsonexample}

\subsection{Vocabulary Mappings}
\label{sec:vocabulary-mappings}

\subsubsection{Schema.org Integration}

BOOST entities map to Schema.org types for web compatibility:

\begin{itemize}
    \item \textbf{Organization} → \texttt{schema:Organization}
    \item \textbf{Transaction} → \texttt{schema:Order}
    \item \textbf{GeographicData} → \texttt{schema:Place}
    \item \textbf{Certificate} → \texttt{schema:Certification}
    \item \textbf{Claim} → \texttt{schema:Claim}
\end{itemize}

\subsubsection{W3C PROV Ontology}

Provenance tracking using PROV vocabulary:

\begin{itemize}
    \item \textbf{MaterialProcessing} → \texttt{prov:Activity}
    \item \textbf{ProcessingHistory} → \texttt{prov:Entity}
    \item \textbf{Operator} → \texttt{prov:Agent}
    \item \textbf{wasGeneratedBy} → \texttt{prov:wasGeneratedBy}
    \item \textbf{wasAttributedTo} → \texttt{prov:wasAttributedTo}
\end{itemize}

\subsubsection{GS1 Vocabulary}

Supply chain standards alignment:

\begin{itemize}
    \item \textbf{productCode} → \texttt{gs1:gtin}
    \item \textbf{locationCode} → \texttt{gs1:gln}
    \item \textbf{shipmentId} → \texttt{gs1:sscc}
    \item \textbf{batchNumber} → \texttt{gs1:batchNumber}
\end{itemize}

\subsection{Entity Context Examples}
\label{sec:entity-context-examples}

\subsubsection{TraceableUnit with Context}

Complete JSON-LD representation of a TraceableUnit:

\begin{jsonexample}{TraceableUnit JSON-LD}
{
  "@context": "https://boost.org/context.jsonld",
  "@type": "biomass:TraceableUnit",
  "@id": "https://example.org/tru/TRU-2025-001",
  
  "traceableUnitId": "TRU-2025-001",
  "unitType": "pile",
  "totalVolume": {
    "@type": "qudt:QuantityValue",
    "qudt:value": 500.0,
    "qudt:unit": "unit:M3"
  },
  "speciesComposition": [{
    "@type": "biomass:SpeciesComponent",
    "species": "Pseudotsuga menziesii",
    "percentage": 75.0
  }],
  "harvestLocation": {
    "@type": "geo:Point",
    "geo:lat": 45.5231,
    "geo:long": -122.6765
  },
  "prov:wasGeneratedBy": {
    "@id": "https://example.org/harvest/HARV-2025-001"
  },
  "prov:wasAttributedTo": {
    "@id": "https://example.org/org/ORG-FOREST-001"
  }
}
\end{jsonexample}

\subsubsection{Transaction with Linked Data}

Transaction linking multiple entities:

\begin{jsonexample}{Linked Transaction}
{
  "@context": "https://boost.org/context.jsonld",
  "@type": "schema:Order",
  "@id": "https://example.org/txn/TXN-2025-001",
  
  "transactionId": "TXN-2025-001",
  "schema:seller": {
    "@id": "https://example.org/org/ORG-SUPPLIER-001"
  },
  "schema:buyer": {
    "@id": "https://example.org/org/ORG-BUYER-001"
  },
  "schema:orderedItem": [{
    "@id": "https://example.org/tru/TRU-2025-001"
  }],
  "schema:price": {
    "@type": "schema:PriceSpecification",
    "schema:price": 85.50,
    "schema:priceCurrency": "USD"
  },
  "prov:startedAtTime": "2025-01-15T09:00:00Z",
  "prov:endedAtTime": "2025-01-15T14:30:00Z"
}
\end{jsonexample}

\subsection{Advanced Features}
\label{sec:jsonld-advanced}

\subsubsection{Named Graphs}

Support for multi-source data using named graphs:

\begin{jsonexample}{Named Graph Structure}
{
  "@context": "https://boost.org/context.jsonld",
  "@graph": [{
    "@id": "https://example.org/graph/supplier",
    "@graph": [
      {
        "@type": "Organization",
        "organizationId": "ORG-001",
        "name": "Forest Products Inc"
      }
    ]
  }, {
    "@id": "https://example.org/graph/certification",
    "@graph": [
      {
        "@type": "Certificate",
        "certificateId": "CERT-FSC-001",
        "issuedTo": {"@id": "ORG-001"}
      }
    ]
  }]
}
\end{jsonexample}

\subsubsection{Framing}

JSON-LD framing for specific data views:

\begin{jsonexample}{Frame Definition}
{
  "@context": "https://boost.org/context.jsonld",
  "@type": "TraceableUnit",
  "harvestedBy": {
    "@type": "Organization",
    "certifications": {
      "@type": "Certificate",
      "certificationType": "FSC"
    }
  }
}
\end{jsonexample}

\subsubsection{Compaction and Expansion}

BOOST supports JSON-LD algorithms:

\begin{itemize}
    \item \textbf{Compaction}: Shortens IRIs using context
    \item \textbf{Expansion}: Expands to full IRIs
    \item \textbf{Flattening}: Creates flat graph structure
    \item \textbf{Normalization}: Canonical RDF representation
\end{itemize}

\subsection{Context Negotiation}
\label{sec:context-negotiation}

\subsubsection{Content Type Headers}

HTTP content negotiation support:

\begin{itemize}
    \item \texttt{application/ld+json} - JSON-LD format
    \item \texttt{application/json} - Plain JSON (context link in header)
    \item \texttt{text/turtle} - RDF Turtle format
    \item \texttt{application/n-quads} - N-Quads format
\end{itemize}

\subsubsection{Profile Parameters}

Profile-based context selection:

\begin{verbatim}
Accept: application/ld+json; 
        profile="https://boost.org/profiles/extended"
\end{verbatim}

\subsection{Implementation Guidance}
\label{sec:jsonld-implementation}

\subsubsection{Python Implementation}

Using PyLD library for JSON-LD processing:

\begin{pythonexample}{JSON-LD Processing}
\begin{minted}[fontsize=\small,linenos=false,breaklines=true]{python}
from pyld import jsonld
import json

# Load BOOST context
with open('boost_context.jsonld') as f:
    context = json.load(f)

# Create entity with context
tru = {
    "@context": context,
    "@type": "TraceableUnit",
    "traceableUnitId": "TRU-001",
    "totalVolume": 100.0
}

# Expand to full IRIs
expanded = jsonld.expand(tru)

# Compact with custom context
compacted = jsonld.compact(expanded, context)

# Convert to RDF
rdf = jsonld.to_rdf(tru)

# Frame for specific view
frame = {"@type": "TraceableUnit"}
framed = jsonld.frame(tru, frame)
\end{minted}
\end{pythonexample}

\subsubsection{JavaScript Implementation}

Browser and Node.js support:

\begin{jsonexample}{JavaScript JSON-LD}
const jsonld = require('jsonld');

// Process BOOST data
async function processBoostData(data) {
  // Add context
  data['@context'] = 'https://boost.org/context.jsonld';
  
  // Validate structure
  const expanded = await jsonld.expand(data);
  
  // Generate RDF
  const nquads = await jsonld.toRDF(data, {format: 'N-Quads'});
  
  return nquads;
}
\end{jsonexample}

\subsection{Semantic Validation}
\label{sec:semantic-validation}

\subsubsection{SHACL Constraints}

Shape validation for semantic correctness:

\begin{jsonexample}{SHACL Shape}
{
  "@context": {"sh": "http://www.w3.org/ns/shacl#"},
  "@type": "sh:NodeShape",
  "sh:targetClass": "biomass:TraceableUnit",
  "sh:property": [{
    "sh:path": "biomass:totalVolume",
    "sh:datatype": "xsd:decimal",
    "sh:minInclusive": 0,
    "sh:maxInclusive": 10000
  }]
}
\end{jsonexample}

\subsubsection{Reasoning and Inference}

Automatic inference capabilities:

\begin{itemize}
    \item Type inheritance from parent classes
    \item Property domain/range validation
    \item Transitive relationship discovery
    \item Consistency checking
\end{itemize}

\subsection{Benefits and Use Cases}
\label{sec:jsonld-benefits}

\subsubsection{Interoperability Benefits}

\begin{itemize}
    \item \textbf{Global Identifiers}: IRIs enable worldwide unique identification
    \item \textbf{Vocabulary Reuse}: Leverage existing ontologies
    \item \textbf{Tool Ecosystem}: Compatible with RDF/SPARQL tools
    \item \textbf{Web Integration}: SEO and knowledge graph inclusion
\end{itemize}

\subsubsection{Supply Chain Use Cases}

\begin{itemize}
    \item \textbf{Cross-Organization Linking}: Connect data across partners
    \item \textbf{Provenance Tracking}: Complete chain of custody
    \item \textbf{Regulatory Reporting}: Machine-readable compliance data
    \item \textbf{Certification Verification}: Linked certificate validation
\end{itemize}

The JSON-LD context provides BOOST with semantic web capabilities essential for modern supply chain interoperability and regulatory compliance.

% ================================
% PART IV: DOMAIN APPLICATIONS
% ================================

% Chapter 11: Regulatory Program Compliance
\section{Regulatory Program Compliance}
\label{sec:regulatory-compliance}
% LCFS Programmatic Reporting and Biofuel Compliance
% Comprehensive documentation for regulatory compliance


The BOOST standard provides comprehensive support for regulatory compliance across multiple biofuel programs, with primary focus on the California Low Carbon Fuel Standard (LCFS). This section documents programmatic reporting workflows, compliance requirements, and implementation guidance for regulatory submissions.

\subsection{Low Carbon Fuel Standard (LCFS) Overview}
\label{sec:lcfs-overview}

The California Low Carbon Fuel Standard, administered by the California Air Resources Board (CARB), is a market-based regulation designed to reduce greenhouse gas emissions from transportation fuels. BOOST provides specialized entities and validation rules to support complete LCFS compliance workflows.

\subsubsection{Regulatory Context}

The LCFS program requires regulated parties to:
\begin{itemize}
    \item Track all fuel transactions with certified pathway attribution
    \item Calculate carbon intensity using CARB-approved methodologies
    \item Submit quarterly reports with complete audit trails
    \item Maintain third-party verification documentation
    \item Demonstrate compliance with sustainability criteria
\end{itemize}

\subsubsection{BOOST's Role in LCFS Compliance}

BOOST enables LCFS compliance through:
\begin{itemize}
    \item \textbf{Pathway Management}: \entity{LcfsPathway} entity for CARB-certified pathways
    \item \textbf{Transaction Tracking}: Enhanced \entity{Transaction} with LCFS-specific fields
    \item \textbf{Quarterly Reporting}: \entity{LcfsReporting} entity for regulatory submissions
    \item \textbf{Credit Calculations}: Automated credit/deficit calculation with validation
    \item \textbf{Audit Trail}: Complete traceability from feedstock to fuel product
\end{itemize}

\subsection{LCFS Entity Integration}
\label{sec:lcfs-entities}

\subsubsection{LcfsPathway Entity}

The \entity{LcfsPathway} entity manages CARB-certified fuel pathways:

\begin{itemize}
    \item \field{pathwayId}: CARB-assigned pathway identifier
    \item \field{pathwayType}: Lookup Table, Tier 1, or Tier 2 pathway
    \item \field{carbonIntensity}: Certified CI value (gCO2e/MJ)
    \item \field{energyEconomyRatio}: EER for credit calculation
    \item \field{certificationDate}: CARB certification date
    \item \field{expirationDate}: Pathway expiration date
    \item \field{verificationStatus}: Active, suspended, or expired
    \item \field{caGreetVersion}: CA-GREET model version used
\end{itemize}

\textbf{Pathway Validation Rules:}
\begin{itemize}
    \item Pathway must be active for transaction date
    \item Carbon intensity must match CARB database
    \item Feedstock must align with pathway specifications
    \item Facility location must match certified production site
\end{itemize}

\subsubsection{Enhanced Transaction Entity}

LCFS-specific transaction fields include:

\begin{itemize}
    \item \field{lcfsPathwayId}: Foreign key to certified pathway
    \item \field{fuelVolume}: Volume in gallons or GGE
    \item \field{fuelCategory}: Fuel type classification
    \item \field{reportingPeriod}: YYYY-Q\# format
    \item \field{regulatedPartyRole}: Producer, importer, blender, or distributor
\end{itemize}

\subsubsection{LcfsReporting Entity}

Quarterly reporting aggregation:

\begin{itemize}
    \item \field{reportingPeriod}: Quarter identifier (e.g., "2025-Q1")
    \item \field{totalFuelVolume}: Aggregate fuel volume
    \item \field{totalCreditsGenerated}: Credits from CI below benchmark
    \item \field{totalDeficitsIncurred}: Deficits from CI above benchmark
    \item \field{netPosition}: Net credit/deficit position
    \item \field{verificationStatus}: Third-party verification status
\end{itemize}

\subsection{Programmatic Reporting Workflows}
\label{sec:lcfs-workflows}

\subsubsection{Quarterly Report Generation}

The quarterly reporting process follows these steps:

\begin{enumerate}
    \item \textbf{Transaction Aggregation}: Collect all transactions for reporting period
    \item \textbf{Pathway Validation}: Verify pathway status and attributes
    \item \textbf{Credit Calculation}: Apply CARB formulas with EER adjustment
    \item \textbf{Report Generation}: Create structured report for submission
    \item \textbf{Verification}: Third-party review if required
    \item \textbf{Submission}: Upload to CARB reporting system
\end{enumerate}

\subsubsection{Credit Calculation Methodology}

LCFS credits are calculated using the formula:

\begin{equation}
\text{Credits} = (\text{Benchmark CI} - \text{Pathway CI}) \times \text{Fuel Volume} \times \text{Energy Density} \times \text{EER} \times 10^{-6}
\end{equation}

Where:
\begin{itemize}
    \item Benchmark CI = CARB-specified carbon intensity target
    \item Pathway CI = Certified pathway carbon intensity
    \item Fuel Volume = Transaction volume in gallons
    \item Energy Density = MJ per gallon for fuel type
    \item EER = Energy Economy Ratio for application
\end{itemize}

\subsubsection{Data Reconciliation Process}

Monthly reconciliation ensures data integrity:

\begin{enumerate}
    \item Compare transaction records with physical inventory
    \item Validate pathway assignments against production records
    \item Cross-check credit calculations with manual verification
    \item Document discrepancies in \entity{DataReconciliation} entity
    \item Generate reconciliation report for audit trail
\end{enumerate}

\subsection{Implementation Examples}
\label{sec:lcfs-examples}

\subsubsection{Renewable Diesel Production Example}

Complete workflow for renewable diesel with forest residue feedstock:

\begin{jsonexample}{LCFS Transaction with Pathway}
\begin{minted}[fontsize=\small,linenos=false,breaklines=true,tabsize=2]{json}
{
  "transactionId": "TXN-LCFS-2025Q1-001",
  "transactionType": "fuel_sale",
  "fuelVolume": 10000,
  "fuelVolumeUnit": "gallons",
  "fuelCategory": "renewable_diesel",
  "lcfsPathwayId": "PATH-CARB-RD-001",
  "reportingPeriod": "2025-Q1",
  "organizationId": "ORG-PACIFIC-001",
  "regulatedPartyRole": "producer",
  "traceableUnitIds": ["TRU-FOREST-001", "TRU-FOREST-002"],
  "carbonIntensity": 35.5,
  "benchmarkCI": 94.17,
  "creditsGenerated": 5862.5
}
\end{minted}
\end{jsonexample}

\subsubsection{Credit Calculation Example}

Using actual CARB values for renewable diesel:

\begin{itemize}
    \item Fuel Volume: 10,000 gallons
    \item Energy Density: 129.65 MJ/gallon (renewable diesel)
    \item Benchmark CI: 94.17 gCO2e/MJ (2025 diesel target)
    \item Pathway CI: 35.50 gCO2e/MJ (certified pathway)
    \item EER: 1.0 (heavy-duty diesel application)
\end{itemize}

\textbf{Calculation:}
\begin{align}
\text{Credits} &= (94.17 - 35.50) \times 10,000 \times 129.65 \times 1.0 \times 10^{-6} \\
&= 58.67 \times 10,000 \times 129.65 \times 10^{-6} \\
&= 7,607 \text{ MT CO2e credits}
\end{align}

\subsection{Multi-Program Compliance Framework}
\label{sec:multi-program}

\subsubsection{Renewable Fuel Standard (RFS) Integration}

BOOST supports RFS compliance through:

\begin{itemize}
    \item RIN generation and tracking capabilities
    \item D-code classification for renewable fuel categories
    \item EPA pathway registration support
    \item Quarterly RFS reporting integration
\end{itemize}

\subsubsection{EU Renewable Energy Directive (RED II) Compliance}

European compliance features include:

\begin{itemize}
    \item GHG savings calculation (minimum 65\% for new facilities)
    \item Sustainability criteria verification
    \item Land use change documentation
    \item Mass balance chain of custody
    \item ISCC certification integration
\end{itemize}

\subsubsection{Regional Program Extensions}

Support for state-level programs:

\begin{itemize}
    \item \textbf{Oregon Clean Fuels Program}: Similar to LCFS with state-specific pathways
    \item \textbf{Washington Clean Fuel Standard}: Launched 2023 with unique requirements
    \item \textbf{British Columbia LCFS}: Provincial program with federal alignment
    \item \textbf{Canada Clean Fuel Regulations}: National program with credit trading
\end{itemize}

\subsection{Data Quality and Compliance}
\label{sec:lcfs-data-quality}

\subsubsection{CARB Data Validation Requirements}

All submissions must meet CARB data quality standards:

\begin{itemize}
    \item \textbf{Completeness}: 100\% transaction coverage required
    \item \textbf{Accuracy}: Volume tolerance ±0.5\%
    \item \textbf{Timeliness}: Quarterly submission within 45 days
    \item \textbf{Consistency}: Cross-period reconciliation required
    \item \textbf{Traceability}: Complete audit trail maintained
\end{itemize}

\subsubsection{Third-Party Verification}

Large regulated entities require annual verification:

\begin{enumerate}
    \item Engage CARB-accredited verification body
    \item Provide access to BOOST data systems
    \item Support site visits and record reviews
    \item Address verification findings
    \item Submit verification statement with annual report
\end{enumerate}

\subsection{Technical Implementation}
\label{sec:lcfs-technical}

\subsubsection{API Endpoints for LCFS Data}

RESTful API design for LCFS operations:

\begin{itemize}
    \item \texttt{GET /lcfs/pathways} - Retrieve active pathways
    \item \texttt{POST /lcfs/transactions} - Submit fuel transaction
    \item \texttt{GET /lcfs/reports/\{period\}} - Generate quarterly report
    \item \texttt{POST /lcfs/credits/calculate} - Calculate credits/deficits
    \item \texttt{GET /lcfs/reconciliation/\{period\}} - Reconciliation report
\end{itemize}

\subsubsection{Automated Report Generation}

Python implementation for quarterly reports:

\begin{pythonexample}{LCFS Report Generation}
\begin{minted}[fontsize=\small,linenos=false,breaklines=true]{python}
from boost_client import create_client
from datetime import datetime

client = create_client()

# Generate Q1 2025 LCFS report
report = client.generate_lcfs_report(
    reporting_period="2025-Q1",
    organization_id="ORG-PACIFIC-001"
)

# Aggregate transactions by pathway
pathway_summary = report.aggregate_by_pathway()

# Calculate total credits/deficits
total_credits = sum(t.credits for t in report.transactions)
total_deficits = sum(t.deficits for t in report.transactions)
net_position = total_credits - total_deficits

# Generate CARB submission format
carb_report = report.format_for_carb_submission()

# Export to required XML format
report.export_to_xml("lcfs_2025_q1_submission.xml")
\end{minted}
\end{pythonexample}

\subsubsection{Error Handling and Validation}

Comprehensive validation before submission:

\begin{itemize}
    \item Pathway expiration checking
    \item Volume balance verification
    \item Credit calculation validation
    \item Duplicate transaction detection
    \item Missing data identification
    \item Format compliance checking
\end{itemize}

\subsection{Regulatory Change Management}
\label{sec:regulatory-changes}

BOOST adapts to regulatory updates through:

\begin{itemize}
    \item \textbf{Schema Versioning}: Track regulatory requirement changes
    \item \textbf{Validation Rule Updates}: Modify business logic for new requirements
    \item \textbf{Backward Compatibility}: Maintain historical data integrity
    \item \textbf{Migration Tools}: Update existing data to new standards
    \item \textbf{Compliance Alerts}: Notify users of regulatory changes
\end{itemize}

This comprehensive framework ensures BOOST implementations maintain full regulatory compliance while adapting to evolving program requirements across multiple jurisdictions.

% Chapter 12: Security Considerations
\section{Security Considerations}
\label{sec:security}
% Security Considerations Section
% Placeholder

\subsection{Data Privacy}
\label{sec:data-privacy}

Implementations \SHOULD{} consider privacy implications of biomass tracking data:

\begin{itemize}
    \item Location data may reveal sensitive commercial information
    \item Biometric identifiers require secure storage and transmission
    \item Personal operator information needs appropriate access controls
\end{itemize}

\subsection{Data Integrity}
\label{sec:data-integrity}

Critical security measures include:

\begin{itemize}
    \item Digital signatures for high-value transactions
    \item Audit trails for all data modifications
    \item Backup and recovery procedures for critical supply chain data
    \item Validation of external data sources and certificates
\end{itemize}

\begin{important}[title=Security Implementation Requirements]
Implementations \MUST{} address authentication of supply chain participants, authorization controls for data access, secure communication channels, and fraud detection mechanisms.
\end{important}

% ================================
% PART V: SUPPORTING MATERIALS
% ================================

% Chapter 13: Examples
\section{Examples}
\label{sec:examples}
% Examples Section
% Placeholder

\subsection{Basic TraceableUnit Example}
\label{sec:basic-tru-example}

\begin{jsonexample}{TraceableUnit JSON-LD Example}
\begin{minted}[
    bgcolor=example!5!white,
    frame=leftline,
    framerule=2pt,
    rulecolor=example!75!black,
    fontsize=\small
]{json}
{
  "@context": "https://boost-standard.org/context.jsonld",
  "@type": "TraceableUnit", 
  "@id": "https://example.com/tru/TRU-001",
  "traceableUnitId": "TRU-FOREST-001",
  "unitType": "pile",
  "uniqueIdentifier": "BIOMETRIC-SIGNATURE-ABC123",
  "totalVolumeM3": 125.5,
  "materialTypeId": "MAT-DOUGLAS-FIR-SAWLOG", 
  "isMultiSpecies": false,
  "harvesterId": "ORG-PACIFIC-FOREST",
  "currentGeographicDataId": "GEO-MILL-YARD-07"
}
\end{minted}
\end{jsonexample}

% Chapter 14: Resources \& Community
\section{Resources \& Community}
\label{sec:resources-community}
% Resources & Community Section
% Placeholder

\subsection{Presentations \& Demonstrations}
\label{sec:presentations}

The BOOST Community Group has developed comprehensive presentations and demonstrations including:

\begin{itemize}
    \item BOOST Kickoff Presentation - Overview of the data standard initiative
    \item Transaction Object Examples - Technical demonstration of data structures
    \item California Agency Engagement presentations for CalRecycle, CDFA, and Department of Conservation
    \item BOOST + LCFS Integration technical presentation
\end{itemize}

\subsection{Community Participation}
\label{sec:community-participation}

\begin{informative}[title=BOOST Membership]
\textbf{Chair:} Peter Tittmann (Carbon Direct)

\textbf{Active Participants:} 15+ members from industry stakeholders, regulatory agencies, certification bodies, and technology providers across the biomass supply chain.
\end{informative}

% ================================
% APPENDICES
% ================================

\appendix

% Appendix A: Entity Relationship Diagrams
\section{Entity Relationship Diagrams}
\label{app:erd}
% Entity Relationship Diagrams Appendix
% Placeholder - will be enhanced with static ERD diagrams

\subsection{Complete Entity Relationship Overview}
\label{sec:complete-erd-overview}

The BOOST data model comprises 33 interconnected entities organized into 7 thematic areas:

\begin{itemize}
    \item \coretraceability{Core Traceability} (5 entities)
    \item \organizational{Organizational Foundation} (6 entities)
    \item \materialsupply{Material \& Supply Chain} (7 entities)
    \item \transaction{Transaction Management} (3 entities)
    \item \sustainability{Measurement \& Verification} (4 entities)
    \item \geographic{Geographic \& Tracking} (2 entities)
    \item \reporting{Compliance \& Reporting} (6 entities)
\end{itemize}

% Static ERD diagrams will be inserted here
% \includegraphics[width=\textwidth]{diagrams/boost-erd-complete.pdf}

\subsection{Thematic Area Diagrams}
\label{sec:thematic-area-diagrams}

% Individual thematic area diagrams will be inserted here

% Appendix B: Complete Entity Reference
\section{Complete Entity Reference}
\label{app:entity-reference}
% Complete Entity Reference Appendix
% Auto-generated from JSON schemas

\subsection{Entity Summary Table}
\label{sec:entity-summary-generated}

\begin{longtable}{@{}p{0.25\textwidth}p{0.15\textwidth}p{0.45\textwidth}p{0.1\textwidth}@{}}
\toprule
\textbf{Entity} & \textbf{Thematic Area} & \textbf{Description} & \textbf{Fields} \\
\midrule
\endfirsthead
\toprule
\textbf{Entity} & \textbf{Thematic Area} & \textbf{Description} & \textbf{Fields} \\
\midrule
\endhead
\bottomrule
\endfoot
Data Reconciliation & Compliance \& Reporting & DataReconciliation entity in BOOST data model & 14 \\
Energy Carbon Data & Compliance \& Reporting & EnergyCarbonData entity in BOOST data model & 18 \\
LCFS Pathway & Compliance \& Reporting & CARB-certified fuel pathway for LCFS compliance with carb... & 15 \\
LCFS Reporting & Compliance \& Reporting & Quarterly LCFS compliance report for regulated entities w... & 18 \\
Mass Balance Account & Compliance \& Reporting & MassBalanceAccount entity in BOOST data model & 8 \\
Product Group & Compliance \& Reporting & ProductGroup entity in BOOST data model & 11 \\
Biometric Identifier & Core Traceability & BiometricIdentifier entity in BOOST data model & 8 \\
Location History & Core Traceability & Historical movement records of TRUs & 14 \\
Material Processing & Core Traceability & Processing operations that transform TRUs with plant part... & 18 \\
Processing History & Core Traceability & Complete timeline of processing events with moisture trac... & 22 \\
Traceable Unit & Core Traceability & Unique biomass tracking unit with BOOST traceability syst... & 21 \\
Geographic Data & Geographic \& Tracking & GeographicData entity in BOOST data model & 10 \\
Tracking Point & Geographic \& Tracking & TrackingPoint entity in BOOST data model & 6 \\
Customer & Material \& Supply Chain & Customer entity in BOOST data model & 4 \\
Equipment & Material \& Supply Chain & Equipment entity representing forestry machinery and equi... & 19 \\
Material & Material \& Supply Chain & Material types and specifications & 15 \\
Species Component & Material \& Supply Chain & Species composition within TRUs & 20 \\
Supplier & Material \& Supply Chain & Supplier entity in BOOST data model & 7 \\
Supply Base & Material \& Supply Chain & SupplyBase entity in BOOST data model & 11 \\
Supply Base Report & Material \& Supply Chain & SupplyBaseReport entity in BOOST data model & 9 \\
BOOST Moisture Content Validation Rules & Measurement \& Verification & Comprehensive validation rules and business logic for moi... & 4 \\
Claim & Measurement \& Verification & Claim entity in BOOST data model & 15 \\
Measurement Record & Measurement \& Verification & Quality measurements and dimensional data & 16 \\
Verification Statement & Measurement \& Verification & VerificationStatement entity in BOOST data model & 5 \\
Audit & Organizational Foundation & Audit entity in BOOST data model & 8 \\
BOOST Operator Entity Validation Schema & Organizational Foundation & Validation schema for personnel and operator management w... & 13 \\
Certificate & Organizational Foundation & Certificate entity representing formal certification reco... & 16 \\
CertificationBody & Organizational Foundation & Certification Body entity representing independent organi... & 11 \\
CertificationScheme & Organizational Foundation & CertificationScheme entity defining certification standar... & 15 \\
Organization & Organizational Foundation & Organization entity with geographic data references and c... & 31 \\
BioRAM Pathway & Other & California BioRAM program pathway for biomass power gener... & 15 \\
BioRAM Reporting & Other & Quarterly BioRAM compliance report for biomass power gene... & 20 \\
Sales Delivery Document & Transaction Management & SalesDeliveryDocument entity in BOOST data model & 11 \\
Transaction & Transaction Management & Transaction entity in BOOST data model & 52 \\
Transaction Batch & Transaction Management & TransactionBatch entity in BOOST data model & 24 \\
\end{longtable}

\subsection{Entity Relationship Map}
\label{sec:entity-relationships-generated}

The following table shows all foreign key relationships between entities:

\begin{longtable}{@{}p{0.2\textwidth}p{0.2\textwidth}p{0.2\textwidth}p{0.3\textwidth}@{}}
\toprule
\textbf{Source Entity} & \textbf{Field} & \textbf{Target Entity} & \textbf{Description} \\
\midrule
\endfirsthead
\toprule
\textbf{Source Entity} & \textbf{Field} & \textbf{Target Entity} & \textbf{Description} \\
\midrule
\endhead
\bottomrule
\endfoot
Audit & \field{auditId} & Audit &  \\
Audit & \field{organizationId} & Organization &  \\
Certificate & \field{certificateId} & Certificate & Standard certificate identifier using... \\
Certificate & \field{OrganizationId} & Organization & Uses EntityNameId convention referenc... \\
Claim & \field{claimId} & Claim &  \\
Customer & \field{customerId} & Customer & Unique identifier for the customer \\
Data Reconciliation & \field{transactionId} & Transaction &  \\
Equipment & \field{equipmentId} & Equipment & Unique identifier for the equipment \\
Equipment & \field{organizationId} & Organization & Foreign key to owning organization \\
Location History & \field{operatorId} & Operator &  \\
Mass Balance Account & \field{organizationId} & Organization &  \\
Material Processing & \field{operatorId} & Operator &  \\
Measurement Record & \field{operatorId} & Operator &  \\
BOOST Operator Entity Validation Schema & \field{operatorId} & Operator & Unique identifier for the operator (P... \\
BOOST Operator Entity Validation Schema & \field{organizationId} & Organization & Employing organization - uses EntityN... \\
Organization & \field{organizationId} & Organization & Unique identifier for the organization \\
Processing History & \field{operatorId} & Operator & Foreign key to operator who performed... \\
Sales Delivery Document & \field{transactionId} & Transaction &  \\
Supplier & \field{supplierId} & Supplier &  \\
Supply Base & \field{OrganizationId} & Organization & Managing organization - uses EntityNa... \\
Supply Base Report & \field{organizationId} & Organization &  \\
Traceable Unit & \field{operatorId} & Operator & Foreign key to operator \\
Tracking Point & \field{operatorId} & Operator &  \\
Transaction & \field{transactionId} & Transaction & Unique identifier for the business tr... \\
Transaction & \field{OrganizationId} & Organization & Primary organization involved in tran... \\
Transaction & \field{CustomerId} & Customer & Customer organization (buyer) - uses ... \\
Transaction Batch & \field{transactionId} & Transaction & Foreign key to parent business transa... \\
Transaction Batch & \field{claimId} & Claim & Foreign key to primary sustainability... \\
\end{longtable}


% Appendix C: JSON Schema Reference
\section{JSON Schema Reference}
\label{app:schema-reference}
% JSON Schema Reference Appendix
% Placeholder - will be populated with schema details

\subsection{Schema Validation Rules}
\label{sec:schema-validation-rules}

This appendix provides detailed information about JSON schema validation rules for all BOOST entities.

% Schema details will be auto-generated from the schema files

\subsection{Context Definitions}
\label{sec:context-definitions}

The BOOST JSON-LD context \MUST{} define:

\begin{itemize}
    \item Semantic mappings for all entity types
    \item Property definitions with appropriate vocabularies
    \item Data type specifications for typed literals
    \item Language specifications for internationalization
\end{itemize}

% Appendix D: Python Reference Implementation
\section{Python Reference Implementation}
\label{app:python-implementation}
% Python Reference Implementation Section
% Comprehensive documentation of the BOOST Python reference implementation

The BOOST standard provides a comprehensive Python reference implementation that demonstrates dynamic, schema-driven data models, validation, and supply chain tracking capabilities for biomass chain of custody operations.

\subsection{Overview}
\label{sec:python-overview}

The Python reference implementation uses a \textbf{dynamic, schema-driven architecture} that automatically adapts to changes in BOOST JSON schemas without requiring code modifications. Key features include:

\begin{itemize}
    \item \textbf{Dynamic Schema-Driven Architecture}: Automatically adapts to schema changes without code modifications
    \item \textbf{Comprehensive Validation}: Schema, business logic, and cross-entity validation with 8 categories of business rules
    \item \textbf{Dynamic Model Generation}: Pydantic models generated directly from JSON schemas at runtime
    \item \textbf{Configuration-Driven Business Rules}: Business logic validation rules defined in configuration files
    \item \textbf{Supply Chain Tracking}: Complete traceability with automatic relationship discovery
    \item \textbf{Multi-Certification Support}: FSC, SBP, PEFC, ISCC, RED II compliance validation
    \item \textbf{Mass Balance Accounting}: Volume and mass conservation validation with configurable tolerance checking
    \item \textbf{JSON-LD Export/Import}: Full semantic web compatibility with schema.org and W3C PROV ontology support
    \item \textbf{Schema Version Compatibility}: Graceful handling of schema evolution and backward compatibility
\end{itemize}

\subsection{Installation}
\label{sec:python-installation}

\subsubsection{Prerequisites}
\label{sec:python-prerequisites}

The Python reference implementation requires:

\begin{itemize}
    \item Python 3.8 or higher
    \item pip package manager
\end{itemize}

\subsubsection{Dependencies}
\label{sec:python-dependencies}

Core dependencies are defined in \code{requirements.txt}:

\begin{pythonexample}{Python Dependencies}
\pythoncode{pydantic>=2.0.0      # Data validation and settings management
jsonschema>=4.0.0    # JSON Schema validation  
requests>=2.28.0     # HTTP library for API calls
pyld>=2.0.0          # JSON-LD processor}
\end{pythonexample}

Installation:

\begin{pythonexample}{Installation Command}
\pythoncode{pip install -r requirements.txt}
\end{pythonexample}

\subsection{Architecture}
\label{sec:python-architecture}

The implementation follows a layered architecture with three main components:

\begin{enumerate}
    \item \textbf{Schema Loader}: Dynamic schema loading and model generation engine
    \item \textbf{Dynamic Validation}: Configuration-driven validation with 8 categories of business rules
    \item \textbf{BOOST Client}: High-level API interface using dynamic models
\end{enumerate}

The architecture ensures automatic adaptation to schema changes while providing comprehensive validation and traceability capabilities.

\subsection{Core Components}
\label{sec:python-core-components}

\subsubsection{SchemaLoader}
\label{sec:python-schema-loader}

The \textbf{SchemaLoader} (\code{schema\_loader.py}) is the foundation component that provides dynamic schema loading and model generation:

\textbf{Key Features:}
\begin{itemize}
    \item \textbf{Automatic Schema Discovery}: Scans directories for \code{validation\_schema.json} files
    \item \textbf{Dynamic Model Generation}: Creates Pydantic models from JSON schemas at runtime
    \item \textbf{Enum Generation}: Dynamically creates Python enums from schema definitions
    \item \textbf{Relationship Discovery}: Analyzes schemas to discover foreign key relationships automatically
    \item \textbf{Primary Key Detection}: Identifies primary key fields from schema patterns
\end{itemize}

\textbf{Usage Example:}

\begin{pythonexample}{SchemaLoader Usage}
\pythoncode{from schema_loader import SchemaLoader

# Initialize with automatic schema discovery
loader = SchemaLoader()

# Get dynamically generated Pydantic models
OrganizationModel = loader.get_model('organization')
TraceableUnitModel = loader.get_model('traceable_unit')

# Get enum values directly from current schemas
org_types = loader.get_field_enum_values('organization', 
                                    'organizationType')
print(f"Available organization types: {org_types}")

# Access relationship information discovered from schemas
relationships = loader.get_relationships('traceable_unit')
primary_key = loader.get_primary_key('organization')}
\end{pythonexample}

\subsubsection{DynamicBOOSTValidator}
\label{sec:python-validator}

The \textbf{DynamicBOOSTValidator} (\code{dynamic\_validation.py}) provides comprehensive, schema-driven validation using configuration-based business rules:

\textbf{Validation Categories:}
\begin{enumerate}
    \item \textbf{Schema Validation}: JSON Schema compliance and structural validation
    \item \textbf{Volume/Mass Conservation}: Physical conservation laws with configurable tolerance checking
    \item \textbf{Temporal Logic}: Date/time consistency rules and processing sequence validation
    \item \textbf{Geographic Logic}: Location-based constraints and transport distance validation
    \item \textbf{Species Composition}: Biological consistency and percentage validation
    \item \textbf{Certification Logic}: Chain of custody validation and certificate integrity
    \item \textbf{Regulatory Compliance}: LCFS, EU RED, and sustainability criteria validation
    \item \textbf{Economic/Quality Logic}: Market constraints and quality assurance validation
\end{enumerate}

\textbf{Usage Example:}

\begin{pythonexample}{DynamicBOOSTValidator Usage}
\pythoncode{from dynamic_validation import DynamicBOOSTValidator

validator = DynamicBOOSTValidator()

# Schema validation against current schema
is_valid, errors = validator.validate_entity("organization", org_data)

# Configuration-driven business logic validation
is_valid, errors = validator.validate_business_logic(
    "material_processing", processing_data)

# Comprehensive cross-entity validation
entities = {
    'organization': [org1, org2],
    'traceable_unit': [tru1, tru2],
    'transaction': [txn1]
}
results = validator.comprehensive_validation(entities)}
\end{pythonexample}

\subsubsection{BOOSTClient}
\label{sec:python-client}

The \textbf{BOOSTClient} (\code{boost\_client.py}) provides a high-level interface that uses the dynamic models and validation system:

\textbf{Core Functions:}
\begin{itemize}
    \item \textbf{Entity Creation}: Create entities using dynamically generated models with automatic validation
    \item \textbf{Schema Introspection}: Query available entities, enums, and constraints from current schemas
    \item \textbf{Supply Chain Analysis}: Trace relationships and analyze supply chains using dynamic models
    \item \textbf{Validation}: Comprehensive validation using all dynamic rules and business logic
    \item \textbf{JSON-LD Support}: Export/import with semantic annotations and context management
\end{itemize}

\textbf{Usage Example:}

\begin{pythonexample}{BOOSTClient Usage}
\pythoncode{from boost_client import create_client

# Initialize client with dynamic schema loading
client = create_client()

# Schema introspection
schema_info = client.get_schema_info()
print(f"Available entities: {schema_info['available_entities']}")

# Dynamic enum discovery
org_types = client.get_available_enum_values('organization', 
                        'organizationType')

# Entity creation with schema validation
org = client.create_organization(
    organization_id="ORG-FOREST-001",
    name="Pacific Forest Products",
    org_type="harvester",  # Validated against current schema
    contact_email="ops@pacificforest.com"
)

# Comprehensive validation
validation = client.validate_all()
if validation['valid']:
    print("✓ All entities pass validation!")}
\end{pythonexample}

\subsection{Dynamic Schema Adaptation}
\label{sec:python-schema-adaptation}

A key strength of the Python implementation is its \textbf{automatic adaptation to schema changes}. Most schema modifications require \textbf{no code changes}.

\subsubsection{Automatically Handled Changes}
\label{sec:python-auto-changes}

\textbf{Adding New Fields:}
\begin{itemize}
    \item New optional fields are immediately available
    \item Required fields trigger validation updates automatically
    \item Default values from schemas are applied automatically
\end{itemize}

\textbf{Adding New Enum Values:}
\begin{itemize}
    \item New enum values become available immediately after schema reload
    \item Validation rules update automatically
    \item No code changes required
\end{itemize}

\textbf{Adding New Entity Types:}
\begin{itemize}
    \item New schema files are discovered automatically
    \item Dynamic models are generated on first access
    \item All validation rules apply automatically
\end{itemize}

\textbf{Modifying Business Logic Rules:}
\begin{itemize}
    \item Configuration file changes are applied automatically
    \item Tolerance values and thresholds update dynamically
    \item Cross-entity validation rules adapt to changes
\end{itemize}

\subsubsection{Schema Change Detection}
\label{sec:python-change-detection}

The system provides built-in tools for schema change management:

\begin{pythonexample}{Schema Change Detection}
\pythoncode{# Check current schema status
client = create_client()
schema_info = client.get_schema_info()

# Validate against current schema
validation = client.validate_all()
if not validation['valid']:
    print("Schema changes detected - validation errors:")
    for error in validation['errors']:
        print(f"  - {error}")

# Refresh schemas after updates
client.refresh_schemas()}
\end{pythonexample}

\subsection{Usage Examples}
\label{sec:python-examples}

\subsubsection{Basic Workflow}
\label{sec:python-basic-workflow}

Complete example demonstrating fundamental BOOST operations:

\begin{pythonexample}{Basic Workflow Example}
\pythoncode{from boost_client import create_client

# Initialize BOOST client
client = create_client()

# Create organizations with schema validation
harvester = client.create_organization(
    organization_id="ORG-001",
    name="Forest Products Inc",
    org_type="harvester",
    contact_email="ops@forestproducts.com"
)

processor = client.create_organization(
    organization_id="ORG-002", 
    name="Sawmill Operations LLC",
    org_type="processor",
    contact_email="info@sawmill.com"
)

# Create traceable units with automatic model generation
log_pile = client.create_traceable_unit(
    traceable_unit_id="TRU-LOGS-001",
    unit_type="pile",
    harvester_id="ORG-001",
    total_volume_m3=125.5,
    sustainability_certification="FSC Mix Credit 70%"
)

# Process materials with conservation validation
lumber = client.create_material_processing(
    processing_id="MP-001",
    input_tru_id="TRU-LOGS-001",
    process_type="sawing",
    processor_id="ORG-002",
    output_volume_m3=95.2  # Validates against conservation rules
)

# Execute transaction with comprehensive validation
transaction = client.create_transaction(
    transaction_id="TXN-001",
    organization_id="ORG-002",
    customer_id="CUST-001",
    transaction_date="2025-08-12",
    quantity_m3=50.0
)

# Comprehensive validation using all dynamic rules
validation = client.validate_all()
if validation['valid']:
    print("✓ All entities validated successfully!")
    
# Export to JSON-LD with semantic annotations
jsonld_output = client.export_to_jsonld(include_context=True)}
\end{pythonexample}

\subsubsection{Certification Management}
\label{sec:python-certification}

Example showing certification claim management:

\begin{pythonexample}{Certification Management Example}
\pythoncode{# Create FSC certified organization
fsc_harvester = client.create_organization(
    organization_id="ORG-FSC-001",
    name="Certified Forest Management",
    org_type="harvester",
    certifications=["FSC-FM/COC-001234"]
)

# Create certified traceable unit
certified_logs = client.create_traceable_unit(
    traceable_unit_id="TRU-FSC-001",
    unit_type="pile",
    harvester_id="ORG-FSC-001",
    total_volume_m3=200.0,
    sustainability_certification="FSC Mix Credit 70%",
    certification_claims=["FSC-FM/COC-001234"]
)

# Validate certification chain integrity
cert_validation = client.validate_certification_chain("TRU-FSC-001")
print(f"Certification valid: {cert_validation['valid']}")}
\end{pythonexample}

\subsubsection{Mass Balance Validation}
\label{sec:python-mass-balance}

Example demonstrating conservation law validation:

\begin{pythonexample}{Mass Balance Validation Example}
\pythoncode{# Multiple input materials
input_tru_1 = client.create_traceable_unit(
    traceable_unit_id="TRU-INPUT-001",
    unit_type="pile", 
    total_volume_m3=100.0
)

input_tru_2 = client.create_traceable_unit(
    traceable_unit_id="TRU-INPUT-002",
    unit_type="pile",
    total_volume_m3=75.0
)

# Processing with multiple inputs
pellet_production = client.create_material_processing(
    processing_id="MP-PELLETS-001",
    input_tru_ids=["TRU-INPUT-001", "TRU-INPUT-002"],
    process_type="pelletizing",
    total_input_volume_m3=175.0,
    total_output_volume_m3=140.0,  # Within tolerance for pelletizing
    efficiency_percent=80.0
)

# Validate mass balance with configurable tolerance
balance_validation = client.validate_mass_balance("MP-PELLETS-001")
print(f"Mass balance valid: {balance_validation['valid']}")
print(f"Efficiency: {balance_validation['efficiency']}%")}
\end{pythonexample}

\subsection{Integration Guidance}
\label{sec:python-integration}

\subsubsection{API Development}
\label{sec:python-api-development}

Using the reference implementation for API development:

\begin{pythonexample}{API Development Example}
\pythoncode{from boost_client import create_client
from flask import Flask, jsonify, request

app = Flask(__name__)
boost_client = create_client()

@app.route('/organizations', methods=['POST'])
def create_organization():
    data = request.json
    try:
        # Dynamic validation using current schema
        org = boost_client.create_organization(**data)
        return jsonify(org.model_dump(by_alias=True))
    except ValueError as e:
        return jsonify({"error": str(e)}), 400

@app.route('/validate/<entity_type>', methods=['POST'])
def validate_entity(entity_type):
    data = request.json
    validation = boost_client.validator.validate_entity(entity_type, data)
    return jsonify({
        "valid": validation[0],
        "errors": validation[1]
    })

# Schema introspection endpoint
@app.route('/schema/info')
def schema_info():
    return jsonify(boost_client.get_schema_info())}
\end{pythonexample}

\subsection{Configuration}
\label{sec:python-configuration}

\subsubsection{Schema Path Configuration}
\label{sec:python-schema-config}

Customize schema loading:

\begin{pythonexample}{Schema Path Configuration}
\pythoncode{# Default: automatic discovery from ../schema/
client = create_client()

# Custom schema path
client = create_client(schema_path="/path/to/boost/schemas")

# Multiple schema sources
loader = SchemaLoader()
loader.add_schema_source("/additional/schemas")}
\end{pythonexample}

\subsubsection{Business Logic Configuration}
\label{sec:python-business-config}

Business logic rules are defined in configuration files:

\begin{jsonexample}{Business Logic Configuration}
\jsoncode{{
  "volumeMassConservation": {
    "materialProcessing": {
      "sawing": {
        "tolerance": 0.05,
        "efficiency_range": [0.7, 0.9]
      },
      "pelletizing": {
        "tolerance": 0.10,
        "efficiency_range": [0.75, 0.85]
      }
    }
  },
  "temporalLogic": {
    "processingWindows": {
      "harvest_to_processing_max_days": 90
    }
  }
}}
\end{jsonexample}

\subsection{Testing and Validation}
\label{sec:python-testing}

\subsubsection{Comprehensive Test Suite}
\label{sec:python-test-suite}

The implementation includes comprehensive tests:

\begin{pythonexample}{Running Tests}
\pythoncode{# Run all tests
python test_enhanced_entities.py

# Test specific validation categories
python -m unittest test_enhanced_entities.TestDynamicValidation.test_mass_balance_validation

# Test schema change robustness
python -m unittest test_enhanced_entities.TestSchemaRobustness}
\end{pythonexample}

\subsubsection{Validation Examples}
\label{sec:python-validation-examples}

Test validation with example data:

\begin{pythonexample}{Validation Examples}
\pythoncode{# Load and validate example data
with open('examples/validation/comprehensive_validation_test_suite.json') as f:
    test_data = json.load(f)

validator = DynamicBOOSTValidator()
results = validator.comprehensive_validation(test_data)

print(f"Validation results: {results['summary']}")
for category, result in results['by_category'].items():
    print(f"  {category}: {'PASS' if result['valid'] else 'FAIL'}")}
\end{pythonexample}

\subsection{Performance Characteristics}
\label{sec:python-performance}

\subsubsection{Initialization Performance}
\label{sec:python-init-performance}

\begin{itemize}
    \item \textbf{Schema Loading}: O(n) where n = number of schema files
    \item \textbf{Model Generation}: O(m) where m = number of entity properties  
    \item \textbf{Caching}: Models cached after first generation for O(1) access
\end{itemize}

\subsubsection{Runtime Performance}
\label{sec:python-runtime-performance}

\begin{itemize}
    \item \textbf{Validation}: O(1) for schema validation, O(r) for relationship validation where r = relationships
    \item \textbf{Entity Creation}: O(1) with cached models
    \item \textbf{Memory Usage}: Moderate (dynamic models cached in memory)
\end{itemize}

\subsubsection{Scalability Considerations}
\label{sec:python-scalability}

\begin{itemize}
    \item \textbf{Large Datasets}: Supports batch validation operations
    \item \textbf{Memory Management}: Efficient caching with configurable limits
    \item \textbf{Concurrent Access}: Thread-safe validation operations
\end{itemize}

\subsection{Standards Compliance}
\label{sec:python-standards}

The Python reference implementation fully supports:

\begin{itemize}
    \item \textbf{BOOST Data Standard} (with automatic adaptation to schema updates)
    \item \textbf{JSON-LD 1.1 Specification}
    \item \textbf{JSON Schema Draft-07}
    \item \textbf{Schema.org Vocabulary} for semantic annotations
    \item \textbf{W3C PROV Ontology} for provenance tracking
\end{itemize}

% ================================
% BACK MATTER
% ================================

% Bibliography
\newpage
\section*{References}
\addcontentsline{toc}{section}{References}

\begin{thebibliography}{99}

% Normative References
\bibitem{RFC2119}
S. Bradner. \textit{Key words for use in RFCs to Indicate Requirement Levels}. RFC 2119, IETF, March 1997. 
\url{https://tools.ietf.org/rfc/rfc2119}

\bibitem{RFC8174}
B. Leiba. \textit{Ambiguity of Uppercase vs Lowercase in RFC 2119 Key Words}. RFC 8174, IETF, May 2017.
\url{https://tools.ietf.org/rfc/rfc8174}

\bibitem{JSON-LD11}
Gregg Kellogg, Pierre-Antoine Champin, Dave Longley. \textit{JSON-LD 1.1}. W3C Recommendation, 16 July 2020.
\url{https://www.w3.org/TR/json-ld11/}

\bibitem{JSON-SCHEMA}
Austin Wright, Henry Andrews. \textit{JSON Schema: A Media Type for Describing JSON Documents}. March 2019.
\url{https://json-schema.org/specification.html}

\bibitem{GEOJSON}
H. Butler, M. Daly, A. Doyle, S. Gillies, S. Hagen, T. Schaub. \textit{The GeoJSON Format}. RFC 7946, IETF, August 2016.
\url{https://tools.ietf.org/rfc/rfc7946}

% Industry Standards
\bibitem{ISO38200}
\textit{Chain of custody of wood and wood-based products}. ISO 38200:2018, International Organization for Standardization, 2018.
\url{https://www.iso.org/standard/69429.html}

\bibitem{SBP-STANDARD-4}
\textit{Chain of Custody Standard}. SBP Standard 4, Version 1.0, Sustainable Biomass Partnership, 2013.
\url{https://sbp-cert.org/documents/standards-documents/}

\bibitem{SBP-STANDARD-5}
\textit{Collection and Communication of Data}. SBP Standard 5, Version 1.0, Sustainable Biomass Partnership, 2013.
\url{https://sbp-cert.org/documents/standards-documents/}

\bibitem{FSC-STD-40-004}
\textit{Chain of Custody Certification}. FSC-STD-40-004, Version 3.0, Forest Stewardship Council, 2017.
\url{https://fsc.org/en/document-centre/documents/resource/392}

\bibitem{PEFC-ST-2002}
\textit{Chain of Custody of Forest Based Products}. PEFC ST 2002:2020, Programme for Endorsement of Forest Certification, 2020.
\url{https://www.pefc.org/standards/chain-of-custody}

% Regulatory References  
\bibitem{CA-LCFS}
\textit{Low Carbon Fuel Standard Regulation}. California Air Resources Board, 2024.
\url{https://ww2.arb.ca.gov/our-work/programs/low-carbon-fuel-standard}

\bibitem{EU-RED-II}
\textit{Renewable Energy Directive II}. Directive (EU) 2018/2001, European Union, 2018.
\url{https://eur-lex.europa.eu/legal-content/EN/TXT/?uri=uriserv:OJ.L_.2018.328.01.0082.01.ENG}

% Technical References
\bibitem{KAULEN-2023}
A. Kaulen, L. Stopfer, K. Lippert, T. Purfürst. \textit{Systematics of Forestry Technology for Tracing the Timber Supply Chain}. 2023.
\url{https://github.com/carbondirect/BOOST/tree/main/references}

\end{thebibliography}

% Acknowledgments
\section*{Acknowledgments}
\addcontentsline{toc}{section}{Acknowledgments}

This specification was developed through the collaborative efforts of the BOOST W3C Community Group with significant contributions from:

\begin{itemize}
    \item \textbf{California Department of Conservation} - Funding and regulatory guidance
    \item \textbf{Forest industry stakeholders} - Requirements analysis and use case development  
    \item \textbf{Certification bodies} - Standards alignment and validation procedures
    \item \textbf{Technology providers} - Implementation guidance and tool development
    \item \textbf{Academic institutions} - Research and analysis support
    \item \textbf{Environmental organizations} - Sustainability criteria and verification methods
\end{itemize}

Special recognition to the contributors of the Interactive ERD Navigator, Python reference implementation, and comprehensive schema validation tools that support this specification.

% Index
\newpage
\printindex

\end{document}