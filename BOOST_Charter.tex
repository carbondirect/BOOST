% Created 2025-05-11 Sun 21:45
% Intended LaTeX compiler: pdflatex
\documentclass[title=small,preset=opensansnote,par=skip]{article}
\usepackage[utf8]{inputenc}
\usepackage{lmodern}
\usepackage[T1]{fontenc}
\usepackage[top=1in, bottom=1.in, left=1in, right=1in]{geometry}
\usepackage{graphicx}
\usepackage{longtable}
\usepackage{float}
\usepackage{wrapfig}
\usepackage{rotating}
\usepackage[normalem]{ulem}
\usepackage{amsmath}
\usepackage{textcomp}
\usepackage{marvosym}
\usepackage{wasysym}
\usepackage{amssymb}
\usepackage{amsmath}
\usepackage[theorems, skins]{tcolorbox}
\usepackage[version=3]{mhchem}
\usepackage[numbers,super,sort&compress]{natbib}
\usepackage{natmove}
\usepackage{url}
\usepackage[cache=false]{minted}
\usepackage[strings]{underscore}
\usepackage[linktocpage,pdfstartview=FitH,colorlinks,
linkcolor=blue,anchorcolor=blue,
citecolor=blue,filecolor=blue,menucolor=blue,urlcolor=blue]{hyperref}
\usepackage{attachfile}
\usepackage{setspace}
\usepackage{minted}
\usepackage{tabularx}
\usepackage{phfnote}
\usepackage{amsmath}
\usepackage{draftwatermark}
\date{\today}
\title{Biomass Open Origin Standard for Tracking (BOOST)\\\medskip
\large Working Group Charter}
\begin{document}

\maketitle

This Charter is work in progress. To submit feedback, please use \href{https://github.com/carbondirect/BOOST}{BOOST repository} Issues where Charter is being developed. 

\begin{itemize}
\item This Charter: \href{https://github.com/carbondirect/BOOST/blob/main/BOOST\_Charter.org}{https://github.com/carbondirect/BOOST/blob/main/BOOST\_Charter.org}
\item Previous Charter: N/A
\item Start Date: \textit{<2025-04-30 Wed>}
\item Last Modified
\phantomsection
\label{}
\begin{verbatim}
2025-05-11
\end{verbatim}
\end{itemize}
\section{Context}
\label{sec:org2eae1f2}

The development of the initial version of this data standard is funded through a grant from the California Department of Conservation. This standard is intended to broadly inform the development of biomass chain of custody tools, with an initial focus on California as the jurisdictional context. California's regulatory environment presents challenges and opportunities for participants in the biomass economy that are broadly transferable to a generalized biomass chain of custody data standard.

The group will seek participation from a balanced range of stakeholders, including civil society organizations, government agencies, small and large businesses, and independent technical experts. Recruitment and engagement efforts will be made to avoid overrepresentation of any single stakeholder group.
\section{Goals}
\label{sec:org1455147}

The mission and goals of this Community Group are to \textbf{develop and} \textbf{maintain a robust data standard for solid biomass}. to provide a structured framework to enable the systematic collection, management, and exchange of information throughout complex supply chains and processes. The goal is to ensure accurate tracking, transparency, and compliance with sustainability criteria, facilitating the reliable transfer of essential information, including sustainability characteristics and data required for greenhouse gas (GHG) emission calculations, from the source to the end-user. This standard is intended to support the needs of organizations involved in the biomass supply chain, agencies and stakeholders in the biomass economy, and systems like Chain of Custody (CoC) software, referencing concepts from documents like the Netherlands Enterprise Agency, \emph{“Guidance Chain of} \emph{Custody: Sustainability Criteria for Solid Biomass for Energy} \emph{Applications.”} and related standards such as those from Sustainable Biomass Program , International Organization for Standardization, Sustainable Forestry Initiative, the Forest Stewardship Council, and the Roundtable on Sustainable Biomaterials
\section{Scope of Work}
\label{sec:org3901de6}

The scope of work includes defining a data standard that specifies the format, structure, and relationships of data elements necessary to document and transfer information about solid biomass for energy applications within a Chain of Custody system.

The standard shall cover the representation of key entities and their associated information as described in relevant guidance and standards, including:

\begin{itemize}
\item \textbf{Organizations (Links)} within the supply chain that take legal ownership of the biomass, including their roles and scope.
\item \textbf{Biomass Consignments} or transactions representing quantities of   biomass transferred or handled, including attributes like quantity,   type, and transaction dates. This includes distinguishing between Incoming and Outgoing Consignments.
\item \textbf{Sustainability Information} linked to consignments, detailing   characteristics based on source and criteria.
\item \textbf{Biomass Categories} (e.g., Category 1-5) based on source and type,   and \textbf{Biomass Types} (e.g., Sustainable, Controlled).
\item Information regarding the \textbf{Biomass Source} and Country of Origin.
\item Data related to \textbf{Mass Balance Accounts} used by organizations to track volumes and manage claims.
\item Information documenting the use of \textbf{Control Methods} like Mass Balance or Physical Separation.
\item Data points necessary for \textbf{GHG emission calculations}, including energy and carbon data, and linking this data to biomass batches. This includes data collected via specific report types such as SAR and SREG.
\item References to \textbf{Certification Schemes} (e.g., FSC, PEFC, SBP) and   associated \textbf{Claims} (e.g., SBP-compliant, SBP-controlled), and how these claims are linked to biomass and transactions.
\item Information related to \textbf{Verification Statements} issued by   independent third parties.
\item Documentation and records used for transactions and registration.
\item Data relevant to a risk-based approach, potentially including elements of a \textbf{Due Diligence System} where applicable, particularly for Category 2 biomass.
\item The standard should consider how to represent the flow of information between suppliers and recipients and the use of documentation like   sales/delivery documents.
\item Consideration of alignment or interoperability requirements with   existing data transfer systems, such as the SBP Data Transfer System  (DTS).
\end{itemize}

The standard should be defined in a technical format suitable for software system implementation, potentially including supporting documentation.
\section{Out of Scope}
\label{sec:org686f2db}

Topics known in advance to be out of scope include:

\begin{itemize}
\item Defining or modifying the sustainability criteria for solid biomass   (these criteria are defined by regulatory or voluntary certification requirements). The   standard will focus on representing data \emph{related} to these criteria. - Defining the specific calculation methodologies for GHG emissions. The standard will define how to represent the data   \emph{required} for these calculations.
\item Defining the processes for conducting audits, certification, or   verification, or the requirements for Certification/Verification   Bodies.
\item Defining the requirements for establishing or operating a complete   Chain of Custody management system (these are defined in CoC standards). The standard will define the   data structure that such systems would manage and exchange.
\item Governing or approving certification schemes or their specific claims.
\item Defining the legal responsibilities of organizations (these are   determined by regulations and contractual agreements).
\end{itemize}
\section{Deliverables}
\label{sec:orge97db6b}

\subsection{Specifications}
\label{sec:orga1e5f46}
The primary deliverable will be a \textbf{Biomass Data Standard} \textbf{specification}. This specification will define the structure, format, and relationships of data elements for solid biomass chain of custody information exchange. The output format (e.g., JSON Schema, XML Schema, OWL/RDF) will be determined by the group. An estimated schedule for key deliverables (e.g., first draft, candidate recommendation) will be developed by the group.
\subsection{Non-Normative Reports}
\label{sec:org6d6f870}

The group may produce other Community Group Reports within the scope of this charter but that are not Specifications, for instance, use cases, requirements documents based on the analysis of sources like the "Guidance" and SBP Standards, or white papers explaining the rationale and implementation considerations for the standard.
\subsection{Test Suites and Other Software}
\label{sec:orgcd83e3d}

The group MAY produce test suites to support the Specifications. These test suites could be used to validate data instances against the defined standard schema. The group MAY also develop reference implementations or libraries to facilitate the adoption and use of the standard. Please see the GitHub LICENSE file for test suite contribution licensing information.
\section{Dependencies or Liaisons}
\label{sec:org15ce070}

The group depends on the requirements and concepts defined in the "Guidance Chain of Custody sustainability criteria for solid biomass for energy applications" and its normative references, including the Verification Protocol and Dutch regulations.

The group should consider potential liaisons or alignment efforts with organizations managing relevant existing standards and systems, including: * \textbf{Sustainable Biomass Program (SBP)}, particularly regarding SBP Standards 4 (CoC), 5 (Data Collection), 6 (GHG Calculation), the SBP Data Transfer System (DTS), and their recognized certification schemes (FSC, PEFC). * \textbf{International Organization for} \textbf{Standardization (ISO)}, specifically regarding ISO 38200:2018 Chain of custody of wood and wood-based products. * Potentially, the \textbf{Netherlands Enterprise Agency (RVO.nl)} and the \textbf{Ministry of} \textbf{Economic Affairs and Climate Policy} in the Netherlands, who commissioned the "Guidance" document and oversee the SDE+ scheme.
\section{Community and Business Group Process}
\label{sec:org5277300}

The group operates under the Community and Business Group Process. Terms in this Charter that conflict with those of the Community and Business Group Process are void. As with other Community Groups, W3C seeks organizational licensing commitments under the W3C Community Contributor License Agreement (CLA). When people request to participate without representing their organization's legal interests, W3C will in general approve those requests for this group with the following understanding: W3C will seek and expect an organizational commitment under the CLA starting with the individual's first request to make a contribution to a group Deliverable. The section on Contribution Mechanics describes how W3C expects to monitor these contribution requests.

The W3C Code of Ethics and Professional Conduct applies to participation in this group.
\section{Work Limited to Charter Scope}
\label{sec:org00d4e0a}

The group will not publish Specifications on topics other than those listed under Specifications above. See below for how to modify the charter.
\section{Contribution Mechanics}
\label{sec:org820306d}

Substantive Contributions to Specifications can only be made by Community Group Participants who have agreed to the W3C Community Contributor License Agreement (CLA). Specifications created in the Community Group must use the W3C Software and Document License. All other documents produced by the group should use that License where possible.

All Contributions are made on the groups public mail list or public contrib list or through contributions in the GitHub repo the group is using for the particular document. This may be in the form of a pull request (preferred), by raising an issue, or by adding a comment to an existing issue. All Github repositories attached to the Community Group must contain a copy of the CONTRIBUTING and LICENSE files.
\section{Transparency}
\label{sec:org9536718}

The group will conduct all of its technical work in public. If the group uses GitHub, all technical work will occur in its GitHub repositories (and not in mailing list discussions). When necessary to facilitate collaboration with members unfamiliar with the use of GitHub, translation of materials in the GitHub repository into google docs can be done. Google docs used for shared editing will be archived and will be accessible to all members. If Google docs are used, all changes in the edited document will be integrated back into GitHub This is to ensure contributions can be tracked through a software tool. Meetings may be restricted to Community Group participants, but a public summary of minutes must be posted to the group's public mailing list, or to a GitHub issue if the group uses GitHub.
\section{Decision Process}
\label{sec:orge508c4f}

This group will seek to make decisions where there is consensus. No single organization, or coalition of organizations with shared commercial interests, shall exercise disproportionate influence over decisions.
\subsection{Initial Proposal Phase}
\label{sec:org37e73ca}

\begin{itemize}
\item Any member can submit a proposal for a new standard or modification
\item Proposals must include technical specifications, rationale, and implementation considerations
\item A discussion period of 1-2 weeks follows each proposal
\end{itemize}
\subsection{Consensus-Seeking Phase}
\label{sec:org183ad9e}

\begin{itemize}
\item Working group facilitator leads structured discussions to identify and resolve concerns
\item All objections must be documented with technical justification
\item Facilitator may request alternative proposals to address objections
\item Consensus is reached when all participants express support or "can live with it"
\end{itemize}
\subsection{Measuring Consensus}
\label{sec:org57a35e8}

\begin{itemize}
\item Facilitator periodically assesses consensus through explicit calls
\item Members indicate: Support, Acceptable, Concerns (but won't block), or Object
\item Consensus is achieved when no members object and at least 75\% actively support or find acceptable
\end{itemize}
\subsection{Fallback Voting Mechanism}
\label{sec:org4c2dd7f}

\begin{itemize}
\item If consensus cannot be reached after reasonable effort (typically 6-8 weeks of discussion):
\begin{itemize}
\item Facilitator calls for a formal vote with 2-week notice
\item At least 2/3 majority (66.7\%) of voting members required for approval
\item All votes must include justification
\item Abstentions are noted but not counted in the percentage calculation
\end{itemize}
\end{itemize}
\subsection{Documentation Requirements}
\label{sec:org783d629}

\begin{itemize}
\item All decisions must document:
\begin{itemize}
\item Whether achieved by consensus or vote
\item Summary of key discussion points and objections
\item Vote counts (if applicable)
\item Minority positions (especially for vote-based decisions)
\end{itemize}
\end{itemize}

After discussion and due consideration of different opinions, a decision should be publicly recorded (where GitHub is used as the resolution of an Issue).

Any decisions reached at any meeting are tentative and should be recorded in a GitHub Issue for groups that use GitHub and otherwise on the group's public mail list. Any group participant may object to a decision reached at an online or in-person meeting within 7 days of publication of the decision provided that they include clear technical reasons for their objection. The Chairs will facilitate discussion to try to resolve the objection according to this decision process. It is the Chairs' responsibility to ensure that the decision process is fair, respects the consensus of the CG, and does not unreasonably favour or discriminate against any group participant or their employer.
\section{Chair Selection}
\label{sec:org22b1576}

Through the release of version 0.1 of the data standard, Carbon Direct represented by Peter Tittmann will serve as Chair. Subsequent to the release of version 0.1 of the standard, participants in this group choose their Chair(s) and can replace their Chair(s) at any time using whatever means they prefer. However, if 5 participants, no two from the same organisation, call for an election, the group must use the following process to replace any current Chair(s) with a new Chair, consulting the Community Development Lead on election operations (e.g., voting infrastructure and using RFC 2777). 1. Participants announce their candidacies. Participants have 14 days to announce their candidacies, but this period ends as soon as all participants have announced their intentions. If there is only one candidate, that person becomes the Chair. If there are two or more candidates, there is a vote. Otherwise, nothing changes. 2. Participants vote. Participants have 21 days to vote for a single candidate, but this period ends as soon as all participants have voted. The individual who receives the most votes, no two from the same organisation, is elected chair. In case of a tie, RFC2777 is used to break the tie. An elected Chair may appoint co-Chairs.

Participants dissatisfied with the outcome of an election may ask the Community Development Lead to intervene. The Community Development Lead, after evaluating the election, may take any action including no action.
\section{Amendments to this Charter}
\label{sec:org164ef57}

The group can decide to work on a proposed amended charter, editing the text using the Decision Process described above. The decision on whether to adopt the amended charter is made by conducting a 30-day vote on the proposed new charter. The new charter, if approved, takes effect on either the proposed date in the charter itself, or 7 days after the result of the election is announced, whichever is later. A new charter must receive 2/3 of the votes cast in the approval vote to pass. The group may make simple corrections to the charter such as deliverable dates by the simpler group decision process rather than this charter amendment process. The group will use the amendment process for any substantive changes to the goals, scope, deliverables, decision process or rules for amending the charter.
\end{document}
